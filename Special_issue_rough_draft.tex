%% 
%% Copyright 2007, 2008, 2009 Elsevier Ltd
%% 
%% This file is part of the 'Elsarticle Bundle'.
%% ---------------------------------------------
%% 
%% It may be distributed under the conditions of the LaTeX Project Public
%% License, either version 1.2 of this license or (at your option) any
%% later version.  The latest version of this license is in
%%    http://www.latex-project.org/lppl.txt
%% and version 1.2 or later is part of all distributions of LaTeX
%% version 1999/12/01 or later.
%% 
%% The list of all files belonging to the 'Elsarticle Bundle' is
%% given in the file `manifest.txt'.
%% 

%% Template article for Elsevier's document class `elsarticle'
%% with numbered style bibliographic references
%% SP 2008/03/01

\documentclass[preprint,review,12pt]{elsarticle}

%% Use the option review to obtain double line spacing
%%\documentclass[authoryear,preprint,review,12pt]{elsarticle}

%% Use the options 1p,twocolumn; 3p; 3p,twocolumn; 5p; or 5p,twocolumn
%% for a journal layout:
%% \documentclass[final,1p,times]{elsarticle}
%% \documentclass[final,1p,times,twocolumn]{elsarticle}
%% \documentclass[final,3p,times]{elsarticle}
%% \documentclass[final,3p,times,twocolumn]{elsarticle}
%% \documentclass[final,5p,times]{elsarticle}
%% \documentclass[final,5p,times,twocolumn]{elsarticle}

%% For including figures, graphicx.sty has been loaded in
%% elsarticle.cls. If you prefer to use the old commands
%% please give \usepackage{epsfig}

%% The amssymb package provides various useful mathematical symbols
\usepackage{amssymb}
%% The amsthm package provides extended theorem environments
%% \usepackage{amsthm}

%% The lineno packages adds line numbers. Start line numbering with
%% \begin{linenumbers}, end it with \end{linenumbers}. Or switch it on
%% for the whole article with \linenumbers.
%% \usepackage{lineno}

\usepackage{fullpage}
\usepackage{amsfonts}
\usepackage{graphicx}
\usepackage{amsmath}
\usepackage{indentfirst}
\usepackage[version=3]{mhchem} % Formula subscripts using \ce{}
\usepackage[T1]{fontenc}       % Use modern font encodings

%%% Old arguments
%\usepackage{graphicx}
%% uncomment according to your operating system:
%% ------------------------------------------------
%\usepackage[latin1]{inputenc}    %% european characters can be used (Windows, old Linux)
%%\usepackage[utf8]{inputenc}     %% european characters can be used (Linux)
%%\usepackage[applemac]{inputenc} %% european characters can be used (Mac OS)
%% ------------------------------------------------
%\usepackage{authblk}
%\usepackage[superscript]{cite}
%\usepackage[document]{ragged2e}
%\usepackage[T1]{fontenc}   %% get hyphenation and accented letters right
%\usepackage{mathptmx}      %% use fitting times fonts also in formulas
%% do not change these lines:
%\pagestyle{empty}                %% no page numbers!
%\usepackage[left=35mm, right=35mm, top=15mm, bottom=20mm, noheadfoot]{geometry}
%%% please don't change geometry settings!
%
%\usepackage{fullpage}
%\usepackage{amsfonts}
%\usepackage{graphicx}
%\usepackage{float}
%\usepackage{amsmath}
%\usepackage{chemfig}
%\usepackage{indentfirst}
%\usepackage{longtable}
%\usepackage{array}
%\usepackage{cellspace}
%\usepackage{palatino}
%%\usepackage{breqn}
%\usepackage{amssymb}
%\usepackage{verbatim}
%\usepackage[colorlinks=true,citecolor=blue,linkcolor=blue]{hyperref}
%\usepackage{siunitx}
%\usepackage{xr}

%% italicized boldface for math (e.g. vectors)
%\newcommand{\bfv}[1]{{\mbox{\boldmath{$#1$}}}}
%% non-italicized boldface for math (e.g. matrices)
%\newcommand{\bfm}[1]{{\bf #1}}          
%
%%\newcommand{\bfm}[1]{{\mbox{\boldmath{$#1$}}}}
%%\newcommand{\bfm}[1]{{\bf #1}}
%\newcommand{\expect}[1]{\left \langle #1 \right \rangle} % <.> for denoting expectations over realizations of an experiment or thermal averages
%
%\newcommand{\var}[1]{{\mathrm var}{(#1)}}
%\newcommand{\x}{\bfv{x}}
%\newcommand{\y}{\bfv{y}}
%\newcommand{\f}{\bfv{f}}
%
%\newcommand{\hatf}{\hat{f}}
%
%\newcommand{\bTheta}{\bfm{\Theta}}
%\newcommand{\btheta}{\bfm{\theta}}
%\newcommand{\bhatf}{\bfm{\hat{f}}}
%\newcommand{\Cov}[1] {\mathrm{cov}\left( #1 \right)}
%\newcommand{\T}{\mathrm{T}}                                % T used in matrix transpose
%
%\newcommand\blfootnote[1]{%
%	\begingroup
%	\renewcommand\thefootnote{}\footnote{#1}%
%	\addtocounter{footnote}{-1}%
%	\endgroup
%}

% The figures are in a figures/ subdirectory.
\graphicspath{{figures/}}

\journal{Fluid Phase Equilibria}

\begin{document}
	
	\begin{frontmatter}
		
		%% Title, authors and addresses
		
		%% use the tnoteref command within \title for footnotes;
		%% use the tnotetext command for theassociated footnote;
		%% use the fnref command within \author or \address for footnotes;
		%% use the fntext command for theassociated footnote;
		%% use the corref command within \author for corresponding author footnotes;
		%% use the cortext command for theassociated footnote;
		%% use the ead command for the email address,
		%% and the form \ead[url] for the home page:
		%% \title{Title\tnoteref{label1}}
		%% \tnotetext[label1]{}
		%% \author{Name\corref{cor1}\fnref{label2}}
		%% \ead{email address}
		%% \ead[url]{home page}
		%% \fntext[label2]{}
		%% \cortext[cor1]{}
		%% \address{Address\fnref{label3}}
		%% \fntext[label3]{}
		
		\title{Improvements and limitations of Mie $n$-6 potential for condensed phase viscosity prediction at saturation and elevated pressures}
		%\title{Improvements and limitations of Mie n-6 force fields for predicting liquid shear viscosity at saturation and elevated pressures}
		
		%% use optional labels to link authors explicitly to addresses:
		%% \author[label1,label2]{}
		%% \address[label1]{}
		%% \address[label2]{}
		
		\author{Richard A. Messerly}
		\ead{richard.messerly@nist.gov}
		\address{Thermodynamics Research Center, National Institute of Standards and Technology, Boulder, Colorado, 80305}
		
		\author{Michelle C. Anderson}
		\ead{michelle.anderson@nist.gov}
		\address{Thermodynamics Research Center, National Institute of Standards and Technology, Boulder, Colorado, 80305}
		
		\author{S. Mostafa Razavi}
		\address{Department of Chemical and Biomolecular Engineering, The University of Akron}
        \ead{sr87@uakron.edu}
		
		\author{J. Richard Elliott}
		\address{Department of Chemical and Biomolecular Engineering, The University of Akron}
		\ead{elliot1@uakron.edu}
		
		%		
		%	\thispagestyle{empty}
		%	%make title bold and 14 pt font (Latex default is non-bold, 16 pt)
		%	\title{\Large \textbf{Transferability of Mie n-6 force fields for predicting liquid shear viscosity at saturation and elevated pressures}}
		%
		%	\date{} % <--- leave date empty
		%	\maketitle\thispagestyle{empty} %% <-- you need this for the first page
		%	\begin{center}
		%		\title{\textbf{ABSTRACT}}\centering{}
		%	\end{center}
		%	\justify
		%	
		%	\author{Richard A. Messerly}
		%	\email{richard.messerly@nist.gov}
		%	\affiliation{Thermodynamics Research Center, National Institute of Standards and Technology, Boulder, Colorado, 80305}
		%	
		%	\author{Michael R. Shirts}
		%	\email{michael.shirts@colorado.edu}
		%	\affiliation{Department of Chemical and Biological Engineering, University of Colorado, Boulder, Colorado, 80309}
		%	
		%	\author{Andrei F. Kazakov}
		%	\email{andrei.kazakov@nist.gov}
		%	\affiliation{Thermodynamics Research Center, National Institute of Standards and Technology, Boulder, Colorado, 80305}
		
		\begin{abstract}
			%% Text of abstract
			%%% MCA abstract:
%			To determining reliable methodologies and models for the estimation of viscosities, equilibrium molecular dynamics simulations for normal and branched alkanes ranging from two to sixteen carbons were performed with the GROMACS package. Viscosities along the liquid/vapor saturation curve and at 293 K and high-pressure conditions were generated using the TraPPE, Potoff, and TAMie force fields. Viscosities were calculated from simulations using the Green-Kubo method. Reliable data and uncertainties were determined by performing many replicate simulations and analyzing the data distributions. Potoff and TAMie, modern force fields making use of the Mie n-6 (the generalized Lennard-Jones 12-6 potential), outperform the older TraPPE force field which makes use of the traditional Lennard-Jones 12-6 potential. Simulations carried out with the Potoff or TAMie potentials more closely follow trends in viscosity. The TraPPE force field consistently under predicts viscosities. Although simulations with the Potoff force field overestimate viscosity with respect to density, a fortuitous cancellation of errors results in good prediction of viscosity with respect to pressure. The performance of the TAMie force field is usually better than TraPPE but slightly worse than Potoff. All force fields perform somewhat better for normal alkanes than for branched alkanes and the differences in performance between force fields is more noticeable in the case of normal alkanes.
		
		%%% JRE abstract:	
%			While many common force fields are developed based only liquid density and heat of vaporization at 25 C, several more recent force fields have taken into account vapor pressure and liquid density over substantial portions of the coexistence curve for multiple compounds simultaneously, enhancing transferability. This manuscript explores the hypothesis that greater accuracy in characterizing the coexistence properties may lead to greater accuracy for viscosity predictions. Three united atom force fields are considered in detail: the TraPPE-UA model of Siepmann and coworkers, the TAMie model of Gross and coworkers, the AUA4 model of Ungerer and coworkers, and the TraMie model of Potoff and coworkers. Equilibrium molecular dynamics simulations are performed in the NVT ensemble using the Green-Kubo method for viscosity characterization. Simulations are performed for linear alkanes with three to twelve carbons and branched alkanes with four to nine carbons.  Simulation conditions follow the saturated liquid from reduced temperatures of 0.5-0.9 and along key isotherms in the dense liquid region. In general, the more accurate force fields for coexistence properties do indeed predict viscosity more accurately. For saturated liquids, both the TraMie and TAMie models provide roughly 10 \% accuracy for linear alkanes, while deviations are closer to 20-30 \% for AUA4 and TraPPE-UA models. For branched alkanes the behavior is more complicated but TraMie still provides roughly 15 \% accuracy while TraPPE-UA adn AUA4 accuracy is around 25-35 \% deviations. TAMie force fields for branched alkanes were unavailable at thee time of this study.For compressed liquids,  the Mie potential models perform better once again, but tend to overestimate the viscosity at very high pressures. Coincidentally, these models also tend to overestimate the pressure, such that plots of viscosity vs. pressure are  accurate to within about 10 \% up to 200 MPa. Experimental viscosity data tend to be sparse above 200 MPa, but accurate predictions are obtained for propane to 1 GPa. Uncertainty estimates increase substantially for high pressures at low reduced temperatures. Nevertheless, a prediction is made for the viscosity of 2,2,4, trimethylhexane at 293K and 1 GPa, in compliance with the guidelines of the 10th IFPSC. In the course of the study, the sensitivity to bonded and non-bonded interactions is observed and several suggestions are made for future force field developments.
			
			While many common force fields are developed based on liquid density and heat of vaporization at room temperature, several more recent force fields have taken into account saturated vapor pressure and saturated liquid density over substantial portions of the vapor-liquid coexistence curve for multiple compounds simultaneously, enhancing transferability. This manuscript explores the hypothesis that greater accuracy in characterizing the coexistence properties may lead to greater accuracy for viscosity predictions. Four united atom force fields are considered in detail: the TraPPE-UA model of Siepmann and coworkers, the TAMie model of Gross and coworkers, the AUA4 model of Ungerer and coworkers, and the TraMie model of Potoff and coworkers. Equilibrium molecular dynamics simulations are performed in the $NVT$ ensemble using the Green-Kubo method for viscosity characterization. Simulations are performed for linear alkanes with two to sixteen carbons and branched alkanes with four to nine carbons. Simulation conditions follow the saturated liquid from reduced temperatures of 0.5 to 0.9 and along the 293 K isotherm in the dense liquid region. 
			
			In general, the more accurate force fields for coexistence properties do indeed predict viscosity more accurately. For saturated liquids, both the TraMie and TAMie models provide roughly 10 \% accuracy for linear alkanes, while deviations are closer to 20 to 50 \% for AUA4 and TraPPE-UA models. For branched alkanes, the behavior is more complicated but TraMie still provides roughly 15 to 20 \% accuracy, while the TAMie force field results in deviations of 20 to 40 \%, and the TraPPE-UA and AUA4 force fields have deviations of approximately 25 to 60 \%. The deviations tend to increase with decreasing temperature, with the exception of the TraMie deviations for propane, which are nearly constant to the triple point temperature. 
			
			For compressed liquids, the Mie potential models perform better once again, but tend to overestimate the viscosity at very high densities. Coincidentally, these models also tend to overestimate the pressure at high densities, such that plots of viscosity with respect to pressure are accurate to within about 10 \% up to 200 MPa. Experimental viscosity data tend to be sparse above 200 MPa, but accurate predictions are obtained for propane to 1 GPa. Uncertainty estimates increase substantially for high pressures at low reduced temperatures. Nevertheless, a prediction is made for the viscosity of 2,2,4-trimethylpentane at 293K and 1 GPa. 
			
			%, in compliance with the guidelines of the 10th IFPSC. In the course of the study, the sensitivity to bonded and non-bonded interactions is observed and several suggestions are made for future force field developments.
			
		\end{abstract}
		
		\begin{keyword}
			%% keywords here, in the form: keyword \sep keyword
			
			%% PACS codes here, in the form: \PACS code \sep code
			
			%% MSC codes here, in the form: \MSC code \sep code
			%% or \MSC[2008] code \sep code (2000 is the default)
			
			Thermophysical Properties \sep Molecular Simulation
			
		\end{keyword}
		
	\end{frontmatter}	
	
%	\section*{Key points}
%	
%	Mie and TAMie potentials are much better at saturation viscosities, despite not being fit directly to them
%	Viscosity density curve is much harder to reproduce
%	Viscosity pressure is adequately predicted with Potoff and TAMie
%	Branched alkanes have slightly worse performance
%	
%	Propane is accurate to nearly 1 GPa
%	Butane agrees more closely with newer REFPROP correlation
%	C12 has similar results for Potoff and TraPPE?
%	
%	Entropy scaling for isooctane?
%	
%	Wrong torsional parameters for some isocompounds?
%	
%	\section*{Outline}
	
	\section{Introduction}
	
	The design of efficient and reliable technical processes requires accurate estimates of thermophysical properties. Shear viscosity $(\eta)$ is an important property for characterizing flow, e.g., sizing pumps, assessing flow assurance in fossil fuel recovery, and lubricating bearings in tribological applications. There are primarily three different means by which shear viscosity estimates are obtained: experimental measurement, semi-empirical prediction models, and molecular simulation (molecular dynamics, MD). Significant limitations exist for each of these methods. 
	
	For example, experimental measurements can be expensive, time-consuming, and challenging at extreme temperatures $(T)$ and pressures $(P)$. Experimental data tend to be distributed among several prototypes of linear, branched, ring, and polar molecules, with many gaps among a homologous series. Most experimental data are available below 200 MPa, while tribological applications may require estimates at pressures as high as 1 GPa. Flow assurance applications are generally at pressures below 200 MPa, but at temperatures of 423 to 523 K. These ever expanding conditions of interest and economic constraints on new measurements foster increased research in predictive methods.
	
	%Many of the oft cited viscosity data were measured before 1950.
	%150 to 250 $^{\rm o}$C
	
%    Although most compounds lack \textit{reliable} experimental data covering a wide range of temperatures, pressures, and densities $(\rho)$,
    
    The National Institute of Standards and Technology (NIST) Reference Fluid Properties (REFPROP) provides ``reference quality'' viscosity correlations for experimentally well-studied compounds (around 100 species). Most compounds do not have sufficient \textit{reliable} experimental data covering a wide range of temperatures, pressures, and densities $(\rho)$ for developing ``reference quality'' correlations. These compounds require predictive methods that pool together data from several related molecular species. 
    
    Semi-empirical prediction models are typically not reliable over the industrially relevant ranges of $P \rho T$ \cite{PGL}. For example, corresponding states methods are recommended for vapors, dense fluids, and high temperature liquids \cite{BLANK}. These methods rely on the similarity of trends in the properties relative to a reference compound, e.g., methane and \textit{n}-octane. Corresponding states methods are less reliable for more complex molecular structures, e.g., branched compounds. Typical compilations indicate that deviations from experiment may vary by 5 to 50 \%, with little guidance about when to expect lower or higher accuracy. For low temperature liquids, group contribution schemes are favored, but these tend to extrapolate poorly when applied to compounds or conditions outside the training set. More recent advances such as machine learning \cite{Mulero2017,Lee2017} and entropy scaling \cite{Lotgering2015} have shown great promise. However, machine learning relies on large amounts of experimental data and extrapolates poorly due to its weak theoretical basis. While entropy scaling has a stronger theoretical basis, it requires a reliable reference viscosity and an adequate equation-of-state, which may not be readily available for the compound of interest.
	
	%  tend to look like a mixed bag, with some methods recommended at high temperatures and others at low temperatures, and still more correlations recommended for pressure effects.
	%   If an effect like branching, for example, alters the trend, the these methods tend to break down.
	
	%Among the recommended semi-empirical methods, many require a known viscosity, like the saturated liquid viscosity, in order to predict the viscosity at unsaturated conditions. For purposes of the current investigation, we have set a goal of predicting the viscosity of 2,2,4 trimethylhexane at 1 GPa. Therefore, neither the saturation viscosity nor the variation with pressure are known. Furthermore, the degree of branching suggests that corresponding states methods may be unreliable.
	
%	Semi-empirical models often struggle from poor extrapolation due to model deficiencies, over-fitting, and the scarcity of \textit{reliable} experimental data over a wide range of $P \rho T$ state space.
	
%	Molecular dynamics requires extremely reliable force fields and robust simulation methods.
	
%	Unfortunately, the range of available (and reliable) experimental viscosity data does not cover the entire range of $P \rho T$ of interest. 
	
	There are two fundamental challenges for utilizing molecular simulation to estimate viscosity. First, obtaining reproducible results is more difficult for transport properties, such as viscosity, than for static properties. Second, viscosity is extremely sensitive to the force field. In addition to the strong dependence on the non-bonded interactions, the bonded potential plays a much greater role for viscosity than for static properties. For example, varying the torsional potential has a significant impact on viscosity, \cite{Nieto2006} while vapor-liquid coexistence is relatively unaffected \cite{Bernard2009}. Therefore, the ability to predict viscosities with molecular simulation requires both robust methods and adequate force fields. 
	
	% JRE1: and varies exponentially along the saturated liquid curve, and more generally with respect to increasing density.
	
	%RAM1: I don't feel like this fits in the molecular simulation discussion unless we mention how the force field must also predict saturation densities and PVT behavior.
	
%    Viscosity estimates can be obtained from both equilibrium molecular dynamics (EMD) and non-equilibrium molecular dynamics (NEMD) simulations. Recently, a ``Best Practices Guide'' for EMD was developed to improve reproducibility. 
	
	Recently, a ``Best Practices Guide'' was developed to address the first challenge, namely, to improve reproducibility \cite{Maginn2018}. We apply the ``Best Practices'' and address some outstanding issues mentioned therein. However, the focus of this study is the second challenge, namely, determining the most accurate force field(s). We investigate the accuracy of united-atom (UA) Mie $n$-6 force fields, a popular class designed for the engineering purpose of predicting thermophysical properties. Specifically, the force fields we compare are the Transferable Potential for Phase Equilibria TraPPE-UA (and TraPPE-UA2), Transferable Anisotropic Mie (TAMie), Potoff, and fourth generation anisotropic-united-atom (AUA4). Recently, it was shown that the Mie 14-6 and 16-6 potentials of TAMie and Potoff, respectively, are overly repulsive at high densities/pressures. The suitability of these force fields for quantitative viscosity prediction has been widely debated in the literature. 
	
%	By contrast, Reference \citenum{Gordon2006} suggests that great improvement is obtained by utilizing a Mie n-6 potential over the traditional Lennard-Jones 12-6 potential. Recently, it was shown that Mie n-6 potentials are overly repulsive at high densities/pressures. For these reasons, we study how well the united-atom Mie n-6 potentials perform both at saturation and elevated pressures.
	
%	However, this study focused on viscosities of saturated liquids.
	
	%Some studies have suggested that united-atom models are not capable of accurately reproducing viscosity and, therefore, anisotropic-united-atom or all-atom models are needed \cite{Allen1997,Payal2012,Mondello1997}.
	
	For example, some studies suggest that, depending on the compound structure and state conditions, united-atom models are inadequate for estimating viscosities and recommend the use of anisotropic-united-atom (AUA) or all-atom (AA) models for this purpose \cite{Allen1997,Payal2012,Mondello1997,Ungerer2007}. However, these studies focused primarily on UA Lennard-Jones (LJ) 12-6 force fields. Reference \citenum{Gordon2006} provides evidence that the UA Mie $n$-6 potential can accurately predict saturated liquid viscosity $(\eta_{\rm liq}^{\rm sat})$ without significant deprecation of other vapor-liquid saturation properties, i.e., saturated liquid density $(\rho_{\rm liq}^{\rm sat})$, saturated vapor density $(\rho_{\rm vap}^{\rm sat})$, and saturated vapor pressure $(P_{\rm vap}^{\rm sat})$. Alternatively, Reference \citenum{Nieto2006} demonstrated significant improvement in viscosity prediction by modifying the AUA4 torsional potential (AUA4m). Considering the significant increase in computational cost of AA simulations, Mie $n$-6 and/or modified torsional potentials are a desirable alternative to AA force fields. 
	
	% of the AUA4 for demonstrated that modifying the torsional potential (AUA4m) that improves viscosity prediction of viscosity. Furthermore, 
	
%	Due to the increased complexity of AUA models and the computational cost of AA models, other methods for improving  
	
%	 it is argued that united-atom or anisotropic-united-atom (AUA) models may be inadequate for predicting viscosity \cite{Ungerer2007}. 
	
%	Reference \citenum{Hoang2017} demonstrates that including viscosity data in the force field development can improve the identification of a unique set of transferable Mie $n$-6 parameters, while improving viscosity predictions simultaneously.
	
%	Reference \citenum{Ungerer2007} discusses different test cases (i.e. state points, compound structures) where united-atom or anisotropic-united-atom models are adequate and inadequate for predicting viscosity.
%	
%    Furthermore, Reference \citenum{Hoang2017} demonstrated that it is important to include viscosity data when parameterizing a Mie n-6 force field to obtain a unique set of transferable parameters.
    
%    The force fields compared in this study were optimized solely with static vapor-liquid coexistence data, e.g., saturated liquid densities and saturated vapor pressures. Therefore, an additional purpose of this study is to determine the transferability of these force fields that were not parameterized without viscosity data.
        
    The force fields compared in this study were optimized solely with vapor-liquid coexistence data, i.e., dynamic properties, such as viscosity, were not included in their parameterization. While the Potoff and TAMie force fields have shown considerable promise in predicting static properties (primarily $P_{\rm vap}^{\rm sat}$), their ability to predict dynamic properties has not been investigated previously. Reference \citenum{Hoang2017} demonstrates that including viscosity data in the force field development can improve the identification of a unique set of transferable Mie $n$-6 parameters, while improving viscosity predictions simultaneously. Notwithstanding the potential benefits of including viscosity as a property of interest during force field development, we assess the accuracy of TraPPE-UA, TAMie, Potoff, and AUA4 for estimating viscosity as they currently stand, including their torsional potential models.  
        
    %One purpose of this study is to determine if the state-of-the-art Mie potentials are adequate of predicting viscosity without modifying the torsional potential.
    
    The outline for the present work is the following. Section \ref{Methods} explains the force fields, simulation methodology, and data analysis. Section \ref{Results} presents the simulation results for each force field, compound, and state point studied. Section \ref{Discussion/Limitations} discusses some important observations and limitations. Section \ref{Conclusions} recaps the primary conclusions from this work.
    
	%it has  does not signific while vapor-liquid coexistence does not depend strongly on the torsional potential,    
	
%	\begin{enumerate}
%		\item Viscosity is an important property for designing chemical systems
%		\item Viscosity data typically do not cover the entire range of $P \rho T$ of interest
%		\item Prediction methods are typically quite poor for viscosity
%		\item Molecular simulation is an attractive alternative, but two main challenges
%		\begin{enumerate}
%			\item Difficulty of obtaining reproducible results from simulation
%			\item Unreliable force fields
%		\end{enumerate}
%		\item This manuscript applies the recent Best Practices to improve reproducibility such that it is possible to elucidate the difference in force fields
%		\item Previous studies have suggested that UA models may be inadequate, while Gordon showed that a Mie potential could accomplish both VLE and viscosity
%		\item This study tests whether the modern Mie potentials that are optimized for saturation thermodynamic properties are transferable to transport properties, e.g. shear viscosity
%	\end{enumerate}
	
	\section{Methods} \label{Methods}
	
	\subsection{Force field} \label{Force Field}
	
	A united-atom (UA) or anisotropic-united-atom (AUA) representation is used for each compound studied, i.e., normal and branched alkanes are represented with CH$_3$, CH$_2$, CH, and C sites. UA models assume that the UA interaction site is that of the carbon atom, while AUA models assume that the AUA interaction site is shifted away from the carbon atom and towards the hydrogen atom(s). Note that TraPPE and Potoff are UA force fields while TraPPE-2, AUA4, and TAMie are AUA force fields. 
	
%	The UA and AUA groups required for normal and branched alkanes are sp$^3$ hybridized CH$_3$, CH$_2$, CH, and C sites. For most literature models, a single (transferable) parameter set is assigned for each interaction site. However, two exceptions exist for the force fields studied. First, TAMie implements a different set of CH$_3$ parameters for ethane. Second, Potoff reports a ``generalized'' and ``short/long'' (S/L) CH and C parameter set. The Potoff ``generalized'' CH and C parameter set is an attempt at a completely transferable set. However, since the ``generalized'' parameters performed poorly for some compounds, the S/L parameter set was proposed, where the ``short'' and ``long'' parameters are implemented when the number of carbons in the backbone is $\le 4$ and $> 4$, respectively. 
	
%	Note that only the terminal CH$_3$ sites are shifted in the TAMie force field. By contrast, AUA4 displaces the interaction location of non-terminal CH$_2$ and CH sites as well. For simplicity, we only utilize the AUA4 force field with compounds that are composed exclusively of CH$_3$ and C interaction sites, i.e., ethane and 2,2-dimethylpropane. Table \ref{tab:bond-lengths} provides the effective bond-lengths for terminal CH$_3$ sites. The bond-length for all non-terminal sites is 0.154 nm.
%	
%	\begin{table}[h!]
%		\caption{Effective bond-lengths in units of nm for terminal (CH$_3$) UA or AUA interaction sites. Empty table entries for TraPPE-2 denote that the force field does not contain the corresponding interaction site type. Empty table entries in AUA4 arise because this force field uses a more complicated construction than the simple effective bond-length approach. Specifically, AUA4 requires CH$_2$ and CH interaction sites that are not along the C-C bond axis.} \label{tab:bond-lengths}
%		\begin{center}
%			\begin{tabular}{|c|c|c|c|c|c|}
%				\hline
%				Bond & TraPPE, Potoff & TAMie & AUA4 & TraPPE-2 \\ \hline
%				CH$_3$-CH$_3$ & 0.154 & 0.194 & 0.1967 & 0.230 \\ 
%				CH$_3$-CH$_2$ & 0.154 & 0.174 & -- & -- \\ 
%				CH$_3$-CH & 0.154 & 0.174 & -- & -- \\
%				CH$_3$-C & 0.154 & 0.174 & 0.1751 & -- \\
%				\hline
%			\end{tabular}
%		\end{center} 
%	\end{table}
%
%    A fixed bond-length is used for each bond between UA or AUA sites. Note that, although static thermodynamic properties are generally insensitive to the choice of fixed or flexible bonds, dynamic properties, such as viscosity, are much more sensitive. For this reason, we test the degree of variability that arises by implementing a harmonic oscillator model. The results are provided as Supporting Information.

%    Although static thermodynamic properties (e.g., $\rho_{\rm liq}^{\rm sat}$) are generally insensitive to the choice of fixed or flexible bonds, dynamic properties (e.g., $\eta$) are much more sensitive. For this reason, we test the degree of variability that arises by implementing a harmonic oscillator model. The results are provided as Supporting Information. 
    
    The results presented in this work utilize fixed bond-lengths, where each bond not involving a CH$_3$ site utilizes a 0.154 nm bond-length. The anisotropic-united-atom models (TAMie, AUA4, and TraPPE-2) use a slightly larger ``effective'' bond-length for CH$_3$ bonds (see Table \ref{tab:bond-lengths}). While the TAMie force field modifies only the terminal CH$_3$ sites, AUA4 displaces the interaction location of CH$_2$ and CH sites as well. For simplicity, we only utilize the AUA4 force field with compounds that are composed exclusively of CH$_3$ and C interaction sites, i.e., ethane and 2,2-dimethylpropane.
    
    \begin{table}[h!]
    	\caption{Effective bond-lengths in units of nm for terminal (CH$_3$) UA or AUA interaction sites. Empty table entries for TraPPE-2 denote that the force field does not contain the corresponding interaction site type. Empty table entries in AUA4 arise because this force field uses a more complicated construction than the simple effective bond-length approach. Specifically, AUA4 requires CH$_2$ and CH interaction sites that are not along the C-C bond axis.} \label{tab:bond-lengths}
    	\begin{center}
    		\begin{tabular}{|c|c|c|c|c|c|}
    			\hline
    			Bond & TraPPE, Potoff & TAMie & AUA4 & TraPPE-2 \\ \hline
    			CH$_3$-CH$_3$ & 0.154 & 0.194 & 0.1967 & 0.230 \\ 
    			CH$_3$-CH$_2$ & 0.154 & 0.174 & -- & -- \\ 
    			CH$_3$-CH & 0.154 & 0.174 & -- & -- \\
    			CH$_3$-C & 0.154 & 0.174 & 0.1751 & -- \\
    			\hline
    		\end{tabular}
    	\end{center} 
    \end{table}
    
    
%    The anisotropic-united-atom models (TAMie, AUA4, and TraPPE-2) use a slightly larger bond-length for CH$_3$ bonds. The ethane bond-length for TAMie, AUA4, and TraPPE-2 are 0.194, 0.1967, and 0.230 nm, respectively. The TAMie bond-lengths are 0.174 nm for all other CH$_3$ sites and the AU4 2,2,-dimethylpropane bond-lengths are 0.1751 nm. For simplicity, we only utilize the AUA4 force field with compounds that are composed exclusively of CH$_3$ and C interaction sites, i.e., ethane and 2,2-dimethylpropane.
    
    Although static thermodynamic properties (e.g., $\rho_{\rm liq}^{\rm sat}$) are generally insensitive to the choice of fixed or flexible bonds, dynamic properties (e.g., $\eta$) are much more sensitive. For this reason, we test the degree of variability that arises by implementing a harmonic oscillator model. The results are provided as Supporting Information.
    
    The same angle and dihedral potentials are used for each force field. Angular bending interactions are evaluated using a harmonic potential:
	\begin{equation}
	u^{\rm bend} = \frac{k_\theta}{2} \left(\theta-\theta_0\right)^2
	\end{equation}
	where $u^{\rm bend}$ is the bending energy, $\theta$ is the instantaneous bond angle, $\theta_0$ is the equilibrium bond angle (see Table \ref{tab:angles}), and $k_\theta$ is the harmonic force constant which is equal to 62500 K/rad$^2$ for all bonding angles. Dihedral torsional interactions are determined using a cosine series:
	\begin{equation}
	u^{\rm tors} = c_0 + c_1 [1+\cos{\phi}] + c_2 [1-\cos{2\phi}] + c_3 [1+\cos{3\phi}]
	\end{equation}
	where $u^{\rm tors}$ is the torsional energy, $\phi$ is the dihedral angle and $c_i$ are the Fourier constants listed in Table \ref{tab:torsions}. 
	\begin{table}[h!]
		\caption{Equilibrium bond angles $(\theta_0)$. $x$ and $y$ are values between 0 and 3.} \label{tab:angles}
		\begin{center}
			\begin{tabular}{|c|c|}
				\hline
				Bending sites & $\theta_0$ (degrees) \\ \hline
				CH$_x$-CH$_2$-CH$_y$ & 114.0 \\ 
				CH$_x$-CH-CH$_y$ & 112.0 \\ 
				CH$_x$-C-CH$_y$ & 109.5 \\  
				\hline
			\end{tabular}
		\end{center} 
	\end{table}
	
	\begin{table}[h!]
		\caption{Fourier constants $(c_i)$ in units of K. $x$ and $y$ are values between 0 and 3.} \label{tab:torsions}
		\begin{center}
			\begin{tabular}{|c|c|c|c|c|}
				\hline
				Torsion sites & $c_0$ & $c_1$ & $c_2$ & $c_3$ \\ \hline
				CH$_x$-CH$_2$-CH$_2$-CH$_y$ & 0.0 & 355.03 & -68.19 & 791.32 \\ 
				CH$_x$-CH$_2$-CH-CH$_y$ & -251.06 & 428.73 & -111.85 & 441.27 \\
				CH$_x$-CH$_2$-C-CH$_y$ & 0.0 & 0.0 & 0.0 & 461.29 \\
				CH$_x$-CH-CH-CH$_y$ & -251.06 & 428.73 & -111.85 & 441.27 \\
				\hline
			\end{tabular}
		\end{center} 
	\end{table}

	Non-bonded interaction energies and forces between sites located in two different molecules or separated by more than three bonds within the same molecule are calculated using a Mie $n$-6 potential (of which the Lennard-Jones, LJ, 12-6 is a subclass) \cite{Herdes2015}:
	\begin{equation} \label{eq:Mie}
	u^{\rm vdw}(\epsilon,\sigma,n;r) = \left(\frac{n}{n - 6}\right)\left(\frac{n}{6}\right)^{\frac{6}{n - 6}} \epsilon \left[\left(\frac{\sigma}{r}\right)^{n} - \left(\frac{\sigma}{r}\right)^6\right]
	\end{equation} 
	where $u^{\rm vdw}$ is the van der Waals interaction, $\sigma$ is the distance $(r)$ where $u^{\rm vdw} = 0$, $-\epsilon$ is the energy of the potential at the minimum $\left(\text{i.e., }u^{\rm vdw} = -\epsilon \text{ and } \frac{\partial u^{\rm vdw}}{\partial r} = 0 \text{ for } r=r_{\rm min} \right)$, and $n$ is the repulsive exponent. The non-bonded Mie $n$-6 force field parameters for TraPPE, TraPPE-2, Potoff, AUA4, and TAMie are provided in Table \ref{tab:nonbonded params}. 
	
	\begin{table}[h!]
		\caption{Non-bonded (intermolecular) parameters for TraPPE \cite{TraPPE,Martin1999} (and TraPPE-2 \cite{TraPPEUA2}), Potoff \cite{Mie,Potoff_branched}, AUA4 \cite{AUA4,Nieto2008}, and TAMie \cite{TAMie,Weidler2016} force fields. The ``short/long'' Potoff CH and C parameters are included in parentheses. The ethane specific parameters for TAMie are included in parentheses.} \label{tab:nonbonded params}
		\begin{center}
			\begin{tabular}{|c|c|c|c|c|c|c|}
				\hline
				\multicolumn{1}{|c}{} & \multicolumn{3}{|c}{TraPPE  (TraPPE-2)} & \multicolumn{3}{|c|}{Potoff (S/L)}  \\ \hline
				United-atom & $\epsilon$ (K) & $\sigma$ (nm) & n & $\epsilon$ (K) & $\sigma$ (nm) & n \\ \hline
				CH$_3$ & 98 (134.5)  & 0.375 (0.352) & 12 & 121.25 & 0.3783 & 16  \\ 
				CH$_2$ & 46 & 0.395 & 12 & 61 & 0.399 & 16 \\ 
				CH & 10 & 0.468 & 12 & 15 (15/14) & 0.46 (0.47/0.47) & 16\\
				C & 0.5 & 0.640 & 12 & 1.2 (1.45/1.2) & 0.61 (0.61/0.62) & 16\\
				\hline
				\multicolumn{1}{|c}{} & \multicolumn{3}{|c}{AUA4} & \multicolumn{3}{|c|}{TAMie} \\ \hline
				CH$_3$ & 120.15  & 0.3607 & 12 & 136.318 (130.780) & 0.36034 (0.36463) & 14 \\ 
				CH$_2$ & 86.29 & 0.3461 & 12 & 52.9133 & 0.40400 & 14 \\ 
				CH & 50.98 & 0.3363 & 12 & 14.5392 & 0.43656 & 14\\
				C & 15.04 & 0.244 & 12 & -- & -- & --\\
				\hline
			\end{tabular}
		\end{center} 
	\end{table}

    Note that TraPPE (TraPPE-2) and TAMie implement an ethane-specific set of CH$_3$ parameters. Also, Potoff reports a ``generalized'' and ``short/long'' (S/L) CH and C parameter set. The ``short'' and ``long'' parameters are implemented when the number of carbons in the backbone is $\le 4$ and $> 4$, respectively. Due to their improved performance, we only provide results for the Potoff S/L parameter set.
    
    %The Potoff ``generalized'' CH and C parameter set is an attempt at a completely transferable set. However, since the ``generalized'' parameters performed poorly for some compounds, the S/L parameter set was proposed, where the ``short'' and ``long'' parameters are implemented when the number of carbons in the backbone is $\le 4$ and $> 4$, respectively.
	
	Non-bonded interactions between two different site types (i.e. cross-interactions) are determined using Lorentz-Berthelot combining rules \cite{Allen1987} for $\epsilon$ and $\sigma$, respectively, and an arithmetic mean for the repulsive exponent $n$ (as recommended in Reference \citenum{Mie}):
	\begin{equation} \label{eq:Lorentz-Berthelot_eps}
	\epsilon_{ij} = \sqrt{\epsilon_{ii} \epsilon_{jj}}
	\end{equation}
	\begin{equation} \label{eq:Lorentz-Berthelot_sig}
	\sigma_{ij} = \frac{\sigma_{ii} + \sigma_{jj}}{2}
	\end{equation}
	\begin{equation} \label{eq:Lorentz-Berthelot_lam}
	n_{ij} = \frac{n_{ii} + n_{jj}}{2}
	\end{equation}
	where the $ij$ subscript refers to cross-interactions and the subscripts $ii$ and $jj$ refer to same-site interactions. 
	
	%\subsection{Force fields}
	%
	%Copy the majority of this section from a previous publication
	%
	%\begin{enumerate}
	%	\item United-atom and AUA models are the focus
	%	\item LJ 12-6 and Mie n-6
	%	\item Four force fields (although some have slightly different ethane parameters)
	%	\item None of these force fields specify bond types, so we used fixed bonds
	%	\item Torsions are the same for each force field
	%\end{enumerate}
	
	\subsection{Simulation set-up}
	
%	in the $NVT$ ensemble (constant number of molecules, $N$, constant volume, $V$, and constant temperature, $T$)
	
	Viscosity estimates can be obtained from both equilibrium molecular dynamics (EMD) and non-equilibrium molecular dynamics (NEMD) simulations. The ``Best Practices Guide'' is currently limited to EMD methods and purports that NEMD might be necessary for high viscosities (greater than 0.02 Pa-s). One purpose of the present work is to demonstrate that, by applying these guidelines, EMD can also provide meaningful estimates for highly viscous systems. 
	
%	 that EMD is capable of providing reproducible estimates of viscosity   recommended for estimating viscosity with the Green-Kubo analysis of EMD. 
	
%	Equilibrium molecular dynamics simulations are performed using GROMACS version 2018 \cite{GROMACS_2018}. Each simulation uses the Velocity Verlet integrator with a 2 fs time-step, Nos{\'e}-Hoover thermostat with a time constant of 1 ps, a cut-off distance for non-bonded interactions with tail corrections for energy and pressure \cite{GROMACS_note}, and fixed bond-lengths constrained using LINCS with a LINCS-order of eight. We implement the non-bonded cut-off distance recommended for each force field, namely, TraPPE, TAMie, and AUA4 utilize a 1.4 nm cut-off while a cut-off of 1.0 nm is employed for Potoff (with the exception of \textit{n}-hexadecane which was unstable with this short cut-off distance). Coulombic interactions are not computed as none of the force fields require partial charges for the compounds studied.   
%	
	
	%Each simulation uses the Velocity Verlet integrator with a 2 fs time-step, Nos{\'e}-Hoover thermostat with a time constant of 1 ps, a cut-off distance for non-bonded interactions with tail corrections for energy and pressure \cite{GROMACS_note}, and fixed bond-lengths constrained using LINCS with a LINCS-order of eight. We implement the non-bonded cut-off distance recommended for each force field, namely, TraPPE, TAMie, and AUA4 utilize a 1.4 nm cut-off while a cut-off of 1.0 nm is employed for Potoff (with the exception of \textit{n}-hexadecane which was unstable with this short cut-off distance). Coulombic interactions are not computed as none of the force fields require partial charges for the compounds studied.   
	
%	The equilibration time is 1 ns, while the production time depends on the system, i.e., the compound and state point, where larger compounds, lower temperatures, and higher densities necessitate longer simulations. For most systems, 1 ns is a sufficient production time, while an 8 ns production time is required for the most viscous systems, e.g., 2,2,4-trimethylpentane at elevated pressures. As recommended by BLANK, we investigate several different production times (1, 2, 4, and 8 ns) for some select systems to verify that our simulations are sufficiently long.
	
	Equilibrium molecular dynamics simulations are performed using GROMACS version 2018 \cite{GROMACS_2018}. GROMACS was compiled using the BLANK compiler and run on a BLANK. Example GROMACS input files (.top, .gro. and .mdp) are provided as Supporting Information. In addition, the shell and python scripts used for preparing and analyzing simulations are available on GitHub \cite{BLANK}. The simulation specifications are provided in Tables \ref{tab:sim_specs} and \ref{tab:thermostats_barostats}.  
	
	\begin{table}[htb!]
		\caption{General simulation specifications.} \label{tab:sim_specs}
		\begin{center}
			\begin{tabular}{|c|c|}
				\hline
				Time-step (fs) & 2 \\
				Equilibration time (ns) & 1 \\
				Production time (ns) & 1, 2, 4, or 8 \\
				Cut-off length (nm) & 1.4 (1.0 for Potoff) \\
				Tail-corrections \cite{GROMACS_note} & $U$ and $P$ \\
				Constrained bonds & LINCS \\
				LINCS-order & 8 \\			     
				Number of molecules & 400 \\
				\hline        
			\end{tabular}
		\end{center}
	\end{table}
	
	\begin{table}[h!]
		\caption{Integrator, thermostat and barostat specifications.} \label{tab:thermostats_barostats}
		\begin{center}
			\begin{tabular}{|c|c|c|c|c|}
				\hline
				& $NPT$ Equil. & $NPT$ Prod. & $NVT$ Equil. & $NVT$ Prod. \\ \hline
				Integrator & Velocity Verlet & Leap frog & Velocity Verlet & Velocity Verlet \\ \hline 
				Thermostat & Velocity rescale & Nos{\'e}-Hoover & Nos{\'e}-Hoover & Nos{\'e}-Hoover \\ \hline 
				Thermostat time-constant (ps) & 1.0 & 1.0 & 1.0 & 1.0 \\ \hline
				Barostat & Berendsen & Parrinello-Rahman & N/A & N/A \\ \hline
				Barostat time-constant (ps) & 1.0 & 5.0 & N/A & N/A \\ \hline
				Barostat compressibility & 4.5e-5 & 4.5e-5 & N/A & N/A \\
				\hline
			\end{tabular}
		\end{center} 
	\end{table}
	
	Note that the non-bonded cut-off distance is 1.4 nm for each force field except Potoff, which employs a 1.0 nm cut-off (as recommended by the authors). For most systems, 1 ns is a sufficient production time, while longer simulations are required for the more viscous systems, e.g., 2,2,4-trimethylpentane at elevated pressures. 
	
	Following ``Best Practices'', we compute $\eta$ with several different production times (1, 2, 4, and 8 ns) for select systems to verify that the results are indistinguishable (see Supporting Information). Furthermore, we investigate system size effects by comparing results with 100, 200, 400, and 800 molecules (see Section \ref{Discussion/Limitations}). In addition, we compare fixed and flexible bonds in the Supporting Information.
	
	%Also, notice that the production time depends on the system, i.e., the compound and state point, where larger compounds, lower temperatures, and higher densities necessitate longer simulations.
	
	%	\begin{table}[h!]
	%		\caption{Thermostat and barostat specifications.} \label{tab:thermostats_barostats}
	%		\begin{center}
	%			\begin{tabular}{|c|c|c|c|c|}
	%				\hline
	%				 & $NPT$ Equil. & $NPT$ Prod. & $NVT$ Equil. & $NVT$ Prod. \\ \hline
	%				Thermostat & Velocity rescale & Nos{\'e}-Hoover & Nos{\'e}-Hoover & Nos{\'e}-Hoover \\ 
	%				Thermostat time-constant (ps) & 1.0 & 1.0 & 1.0 & 1.0 \\
	%				Barostat & Berendsen & Parrinello-Rahman & N/A & N/A \\
	%				Barostat time-constant (ps) & 1.0 & 5.0 & N/A & N/A \\
	%				Barostat compressibility & 4.5e-5 & 4.5e-5 & N/A & N/A \\
	%				\hline
	%			\end{tabular}
	%		\end{center} 
	%	\end{table}
	
%	\begin{table}[htb!]
%		\caption{General simulation specifications.} \label{tab:sim_specs}
%		\begin{center}
%			\begin{tabular}{|c|c|}
%				\hline
%				Integrator & Velocity Verlet \\
%				Time-step (fs) & 2 \\
%				Equilibration time (ns) & 1 \\
%				Production time (ns) & 1, 2, 4, or 8 \\
%				Cut-off length (nm) & 1.4 (1.0 Potoff) \\
%				Tail-corrections & Energy and Pressure \cite{GROMACS_note} \\
%				Constraints & LINCS \\
%				LINCS-order & 8 \\			     
%				\hline        
%			\end{tabular}
%		\end{center}
%	\end{table}

	When the viscosity is desired at a prescribed temperature and density $(\eta(\rho,T))$, three stages are required: energy minimization, $NVT$ equilibration, and $NVT$ production. When the viscosity is desired at a prescribed temperature and pressure $(\eta(P,T))$, five stages are required: energy minimization, $NPT$ equilibration, $NPT$ production, $NVT$ equilibration, and $NVT$ production. Note that, according to ``Best Practices'', the final production stage simulations are always performed using the $NVT$ ensemble. 
	
	As recommended by ``Best Practices,'' we utilize 30 to 60 independent replicates to improve the precision and to provide more rigorous estimates of uncertainty. To ensure independence between replicates, the entire series of MD simulations are repeated for each replicate. 
	
	% (constant number of molecules, $N$, constant volume, $V$, and constant temperature, $T$). 
	
	%, 30 to 60 independent replicate simulations are performed for each system
	
	%This is the logical choice when the viscosity is desired at a given temperature and density but not  By contrast, when the viscosity is desired at a prescribed pressure, 
	
%	As recommended in Reference BLANK, we investigate system size effects by comparing results with 100, 200, 400, and 800 molecules. This analysis is provided as Supporting Information. 
%	
%	Example input files are provided as Supporting Information.
	
	%  A system size of 400 molecules is used for ethane, propane, and \textit{n}-butane, while all other compounds use 800 molecules. 
	
	%$(\eta_{P > P_{\rm vap}^{\rm sat}}^{\rm 293 K})$.
		
	Two different classes of viscosity are investigated in this study, namely, saturated liquid viscosity $(\eta_{\rm liq}^{\rm sat})$ and compressed liquid viscosities at a temperature of 293 K $(\eta_{\rm liq}^{\rm comp})$. Saturated liquid viscosities are estimated by performing $NVT$ ensemble simulations at the saturation temperature $(T^{\rm sat})$ and saturated liquid density $(\rho_{\rm liq}^{\rm sat})$. The simulation densities correspond to the REFPROP $\rho_{\rm liq}^{\rm sat}$, which is admittedly not necessarily the same as the force field $\rho_{\rm liq}^{\rm sat}$. This point is discussed in greater detail in Section \ref{Discussion/Limitations}. 
	
%	There are at least three reasons why we perform simulations at the REFPROP $\rho_{\rm liq}^{\rm sat}$ instead of the force field $\rho_{\rm liq}^{\rm sat}$. First, this approach allows for a fair comparison of the force fields' ability to predict viscosity, without penalizing force fields which are less accurate at predicting $\rho_{\rm liq}^{\rm sat}$ or rewarding force fields that mask their deficiencies in predicting viscosity by over- or under-estimating $\rho_{\rm liq}^{\rm sat}$. Second, since each of the studied force fields utilized $\rho_{\rm liq}^{\rm sat}$ data in their optimization, deviations between the REFPROP and force field values are small, typically less than 1 \%. However, small differences in density have been reported to result in large differences in viscosity. For this reason, a small set of validation simulations are performed to determine the variability caused by utilizing the REFPROP densities. The force field saturated liquid densities were obtained from the literature.      
%	
%	The use of REFPROP $\rho_{\rm liq}^{\rm sat}$ caused some simulations to be in a meta-stable state. Specifically, this occurs when the force field vapor pressure is less than the REFPROP vapor pressure. Fortunately, this is uncommon as Potoff, TAMie, and AUA4 are quite reliable for estimating $P_{\rm vap}^{\rm sat}$ and TraPPE significantly over-estimates $P_{\rm vap}^{\rm sat}$.
	
	Two different simulation protocols are implemented for estimating compressed liquid viscosities $(\eta_{\rm liq}^{\rm comp})$. Specifically, we perform simulations with each force field either at the same $\rho$ or the same $P$. For the purpose of comparing trends between force fields and REFPROP, these two methods are essentially equivalent. From a practical standpoint, estimating $\eta$ at a given $P$ requires performing preliminary $NPT$ ensemble simulations to determine the corresponding box size.
	
	% Since comparing force fields at the same density does not require a preliminary $NPT$ simulation to determine the box size, this approach has a small computational benefit. There is a small computational benefit t is computationally less expensive to perform simulations at the same densities, since this does not require a preliminary NPT there is no clear advantage for either approach. 
	
%	The $\eta_{\rm liq}^{\rm sat}$
%	 
%	Estimates for viscosity are obtained alo
	
%	Molecular dynamics simulations for this study are performed in the $NVT$ ensemble (constant number of molecules, $N$, constant volume, $V$, and constant temperature, $T$) using GROMACS version 2018 \cite{GROMACS_2018}. Each simulation uses the Velocity Verlet integrator with a 2 fs time-step, 1.4 nm cut-off for non-bonded interactions with tail corrections for energy and pressure \cite{GROMACS_note}, Nos{\'e}-Hoover thermostat with a time constant of 1 ps, and fixed bond-lengths constrained using LINCS with a LINCS-order of eight. Coulombic interactions are not computed as none of the force fields require partial charges for the compounds studied. The equilibration time is 0.1 ns for ethane and propane, 0.2 ns for \textit{n}-butane, and 0.5 ns for all other compounds. The production time is 1 ns for ethane, 2 ns for propane and \textit{n}-butane, and 4 ns for all other compounds. Replicate simulations are performed for \textit{n}-octane to validate that a single MD run of this length agrees with the average of several replicates, to within the combined uncertainty. A system size of 400 molecules is used for ethane, propane, and \textit{n}-butane, while all other compounds use 800 molecules. Example input files are provided as Supporting Information.
	
%	\begin{enumerate}
%		\item Two types of simulations performed, saturation and 293 K for compressed systems
%		\item Saturation simulations use the REFPROP densities such that, in some cases, the force field is actually in a metastable state
%		\item Performed some simulations at reported saturation conditions
%		\item NPT performed for each replicate such that a distribution of box sizes is obtained
%		\item Depending on the system, a simulation of 1, 2, 4, or 8 ns was used for the production stage
%		\item Details are in supporting information
%	\end{enumerate}
	
	\subsection{Data analysis}
	
	Following the ``Best Practices'' recommendation, we implement the Green-Kubo ``time-decomposition'' analysis to extract viscosity from EMD simulations. We refer the interested reader to References \citenum{Maginn2018} and \citenum{Zhang2015} for further details. In brief, the Green-Kubo integral is computed with respect to time according to
%	\begin{equation} \label{eq:Green_Kubo}
%	\frac{V}{3 k_{\rm B} T N_{\rm reps}} \sum_{n=1}^{N_{\rm reps}} \sum_{\alpha \ne \beta} \int_{0}^{\infty}dt\left\langle \tau_{\alpha\beta,n}(t) \tau_{\alpha\beta,n}(0)\right\rangle_{t_0}
%	\end{equation} 
	\begin{equation} \label{eq:Green_Kubo}
	\eta(t) = \frac{V}{3 k_{\rm B} T N_{\rm reps}} \sum_{n=1}^{N_{\rm reps}} \sum_{\alpha \ne \beta} \int_{0}^{t}dt'\left\langle \tau_{\alpha\beta,n}(t') \tau_{\alpha\beta,n}(0)\right\rangle_{t_0}
	\end{equation} 
	where $V$ is the volume, $k_{\rm B}$ is the Boltzmann constant, $\langle \cdots \rangle_{t_0}$ denotes an average over time origins, $\alpha$ and $\beta = x, y, $ or $z$ Cartesian coordinates, and $\tau_{\alpha\beta,n}$ is the $\alpha$-$\beta$ off-diagonal stress tensor element for the $n^{\rm th}$ replicate. 
	
	$\tau_{\alpha\beta,n}$ is recorded every 6 fs (3 time-steps) to adequately integrate the initial rapid decay of the autocorrelation function. To improve precision, Equation \ref{eq:Green_Kubo} averages several (between 30 and 60) independent replicate simulations $(N_{\rm reps})$, twelve different time-origins $(t_0)$, and all three unique off-diagonal components of $\tau$ (hence the factor of 3 in the denominator of Equation \ref{eq:Green_Kubo}).
	
	%	Since Equation \ref{eq:Green_Kubo} requires information as $t \rightarrow \infty$ and the ``running integral'' can become quite noisy at long times, it is important to fit the ``running integral'' to a function that can be extrapolated to the ``true'' infinite time limit $(\eta^{\infty})$. Per ``Best Practices'', we use a double-exponential function for this purpose
	
	The ``true'' viscosity is obtained by evaluating Equation \ref{eq:Green_Kubo} in the infinite-time-limit, i.e., as $t \rightarrow \infty$. However, the long-time tail of the Green-Kubo integral is often quite noisy and does not converge nicely. For this purpose, we fit a double-exponential function to the ``running integral''
	\begin{equation} \label{eq: Double exponential}
	\eta(t) = A \alpha \tau_1 \left(1-\exp{(-t/\tau_1)}\right) + A (1-\alpha) \tau_2 \left(1-\exp{(-t/\tau_2)}\right)
	\end{equation}
	where $A, \alpha, \tau_1, $ and $\tau_2$ are fitting parameters and $\eta^\infty = A \alpha \tau_1 + A (1-\alpha) \tau_2$ is the infinite-time-limit viscosity. 
	
%	Obtaining the ``true'' infinite time limit viscosity $(\eta^{\infty})$ requires evaluating Equation \ref{eq:Green_Kubo} as $t \rightarrow \infty$. Following ``Best Practices'', we fit a double-exponential function to the ``running integral''
%	\begin{equation} \label{eq: Double exponential}
%	\eta(t) = A \alpha \tau_1 \left(1-\exp{(-t/\tau_1)}\right) + A (1-\alpha) \tau_2 \left(1-\exp{(-t/\tau_2)}\right)
%	\end{equation}
%	where $A, \alpha, \tau_1, $ and $\tau_2$ are fitting parameters and $\eta^\infty = A \alpha \tau_1 + A (1-\alpha) \tau_2$.
	
	Since the Green-Kubo ``running integral'' suffers from extreme fluctuations at long times, Equation \ref{eq: Double exponential} is fit by minimizing a weighted sum-squared error objective function. Weights are equal to the inverse of the standard deviation $(\sigma_{\eta})$ of the replicate simulations. The dime dependence of $\sigma_{\eta}$ is modeled with $A t^{b}$, where $A$ and $b$ are fitting parameters.
	
	Following a heuristic proposed in Reference \citenum{Zhang2015}, data are excluded where $\sigma_{\eta} > 0.4 \times \eta^{\infty}$. Occasionally this heuristic resulted in a cut-off that was too short, which lead to very poor fits. In such cases, it was necessary to modify the heuristic to $0.8 \times \eta^{\infty}$. 
	
	Erroneously large fluctuations also exist at very short times. Following ``Best Practices,'' only data for $t > 3$ ps are included in the fitting of Equation \ref{eq: Double exponential}. 
	 
%	To account for the increasing fluctuations in the ``running integral'' with respect to time, Equation \ref{eq: Double exponential} is fit by minimizing a weighted sum-squared error objective function. The weight model, $A t^{-b}$, is fit to the standard deviation $(\sigma_{\eta})$ of the replicate simulations. In addition, following a heuristic proposed by Zhang et al., data are excluded where $\sigma_{\eta} > 40$ \% $\eta^{\infty}$. Occasionally this heuristic resulted in a cut-off time that was too short, which lead to very poor fits. In such cases, it was necessary to manually increase the cut-off time. Large fluctuations also exist at very short times. Following ``Best Practices,'' only data for $t > 3$ ps are included in the fitting of Equation \ref{eq: Double exponential}. 
	
	As recommend by ``Best Practices'', uncertainties are obtained by bootstrap re-sampling. Specifically, the fitting process described previously is repeated hundreds of times using randomly selected subsets of replicate simulations. Furthermore, each repetition uses a randomly selected long-time cut-off to account for the uncertainty in the \% heuristic. A 95 \% confidence interval is obtained from the distribution of bootstrap estimates for $\eta^\infty$. An example of this process is provided as Supporting Information.
	
%	, and  are employed for a single MD run. The average of 30 to 60 independent replicate simulations $(N_{\rm reps})$ is fit to a double-exponential function by minimizing a weighted sum-squared error objective function.  
	
%	Following the ``Best Practices'' recommendation, we implement the Green-Kubo ``time-decomposition'' analysis to extract viscosity from EMD simulations. We refer the interested reader to References BLANK and BLANK for further details. In brief, the Green-Kubo integral is computed with respect to time for each independent MD simulation from the three off-diagonal components of the stress tensor
%	\begin{equation}
%	\frac{V}{3 k_{\rm B} T N_{\rm reps}} \sum_{n=1}^{N_{\rm reps}} \sum_{\alpha \ne \beta} \int_{0}^{\infty}dt\left\langle \tau_{\alpha\beta,n}(t) \tau_{\alpha\beta,n}(0)\right\rangle_{t_0}
%	\end{equation} 
%	where $V$ is the volume, $k_{\rm B}$ is the Boltzmann constant, $\langle \cdots \rangle_{t_0}$ denotes an average over time origins, $\alpha$ and $\beta = x, y, $ or $z$ Cartesian coordinates, and $\tau_{\alpha\beta,n}$ is the $\alpha$-$\beta$ off-diagonal stress tensor element for the $n^{\rm th}$ replicate. 
	
%	$\tau_{\alpha\beta,n}$ is recorded every 6 fs (3 time-steps) to adequately integrate the initial rapid decay of the autocorrelation function. Twelve different time-origins $(t_0)$ are employed for a single MD run. The average of 30 to 60 independent replicate simulations $(N_{\rm reps})$ is fit to a double-exponential function by minimizing a weighted sum-squared error objective function. The weighting model, $A t^{-b}$, is fit to the standard deviation $(\sigma_{\eta})$ of the replicate simulations. Data are excluded at very short times, less than 3 ps, and at long times, when $\eta_{\sigma} > 40$ \% $\eta^{\infty}$. Uncertainties are obtained by bootstrap re-sampling replicate simulations and by varying the time cut-off between 30 \% and 50 \%. 
	
	%In addition, each bootstrap re-sampling utilizes a random cut-off time of 30 \% - 50 \%.
	
%	According to the recommendation found in Reference BLANK, we utilize the Green-Kubo ``time-decomposition'' method proposed by Zhang et al. We refer the interested reader to References BLANK and BLANK for further details. In brief, the Green-Kubo integral is computed with respect to time for each independent MD simulation. Twelve different time-origins are employed for a single MD run. The average of 30 to 60 independent replicate simulations is fit to a double-exponential function using a weighted sum-squared error approach. The weighting model, $A t^{-b}$, is fit to the standard deviation $(\sigma_{\eta})$ of the replicate simulations. As recommended, data are excluded at very short times, less than 3 ps, and at long times, where $\eta_{\sigma} > 40$ \% $\eta^{\infty}$.  
	
	  
	
%	 The fit is    simul 12 different time-origins.
%	
%	The average Green-Kubo integral is obtained 
%	
%	A double-exponential function is fit to the average of 30 to 60 replicat
%	
%	In brief, the algorithm for obtaining $\eta$ is
%	
%	\begin{enumerate}
%		\item Divide a single MD simulation into 12 equal time blocks (i.e., 12 different time-origins)
%		\item Average 
%	\end{enumerate}
%	
%	 $\eta$ is estimated by fitting a double-exponential function to the average Green-Kubo viscosit
%	
%	Refer to Best Practices document
%	
%	\begin{enumerate}
%		\item Use 40\% sigma for cut-off
%		\item Fit sigma to power model
%		\item Fit viscosity to double exponential
%		\item Bootstrap uncertainties by resampling replicate simulations
%		\item 12 time origins
%	\end{enumerate}
%	
%	We have tried to 
	
	\section{Results} \label{Results}
	
	Seven normal and seven branched alkanes of varying chain-length and degree of branching are simulated in this study. We only consider compounds with available REFPROP equations-of-state and viscosity correlations \cite{LEMMON-RP91}. Specifically, we simulate ethane \cite{Ethane2006,Vogel2015}, propane \cite{Propane2009,Vogel2016}, \textit{n}-butane \cite{Butane2006,Hermann2018}, \textit{n}-octane \cite{Beckmueller2017,Huber2004FPE}, \textit{n}-dodecane \cite{Lemmon2004,Huber2004}, \textit{n}-hexadecane \cite{Romeo2018,Vesovic2017}, \textit{n}-docosane \cite{Romeo2018,Huber2018}, 2-methylpropane \cite{Lemmon2006,Vogel2000}, 2-methylbutane \cite{Lemmon2006,Huber2018}, 2-methylpentane \cite{Lemmon2006,Huber2018}, 3-methylpentane \cite{Gao2017,Huber2018}, 2,2-dimethylpropane \cite{Lemmon2006,Huber2018}, 2,3-dimethylbutane \cite{Gao2017,Huber2018}, and 2,2,4-trimethylpentane \cite{Blackham2017,Huber2018}. 
	
	Each compound was simulated using the TraPPE (UA LJ 12-6) and Potoff S/L (UA Mie 16-6) force fields. Potoff ``short'' parameters are used for 2-methylpropane, 2-methylbutane, 2,2-dimethylpropane, and 2,3-dimethylbutane while Potoff ``long'' parameters are utilized for 2-methylpentane, 3-methylpentane, and 2,2,4-trimethylpentane. 2,2-dimethylpropane and 2,2,4-trimethylpentane were not simulated using the TAMie (AUA Mie 14-6) force field since we are not aware of any TAMie parameters for C sites. Only ethane and 2,2-dimethylpropane were simulated with AUA4 (AUA LJ 12-6). 
	
	Table \ref{tab:simulations_performed} demonstrates which compounds, force fields, and viscosity types were simulated in this study.
	
	\begin{table}[h!]
		\caption{Compounds, force fields, and state points. ``X'': simulated, ``O'': not simulated, ``S'' simulated with ``Short'' parameters, ``L'' simulated with ``Long'' parameters.} \label{tab:simulations_performed}
		\begin{center}
			\begin{tabular}{|c|c|c|c|c|c|c|c|c|}
				\hline
				\multicolumn{1}{|c}{} & \multicolumn{2}{|c}{TraPPE (TraPPE-2)} & \multicolumn{2}{|c|}{Potoff (S/L)} & \multicolumn{2}{|c}{AUA4} & \multicolumn{2}{|c|}{TAMie}  \\ \hline
				Compound & $\eta_{\rm liq}^{\rm sat}$ & $\eta_{\rm liq}^{\rm comp}$ & $\eta_{\rm liq}^{\rm sat}$ & $\eta_{\rm liq}^{\rm comp}$ & $\eta_{\rm liq}^{\rm sat}$ & $\eta_{\rm liq}^{\rm comp}$ & $\eta_{\rm liq}^{\rm sat}$ & $\eta_{\rm liq}^{\rm comp}$ \\ \hline
				ethane & X & X & X & X & X & X & X & X \\ \hline
				propane & X & X & X & X & O & O & X & X \\ \hline
				\textit{n}-butane & X & X & X & X & O & O & X & X \\ \hline
				\textit{n}-octane & X & X & X & X & O & O & X & X \\ \hline
				\textit{n}-dodecane & X & O & X & O & O & O & X & O \\ \hline
				\textit{n}-hexadecane & X & O & X & O & O & O & X & O \\ \hline
				2-methylpropane & X & X & S & S & O & O & X & X \\ \hline
				2-methylbutane & X & X & S & S & O & O & X & X \\ \hline
				2,2-dimethylpropane & X & X & S & S & X & X & O & O \\ \hline
				2,3-dimethylbutane & X & X & S & S & O & O & X & X \\ \hline
				2-methylpentane & X & X & L & L & O & O & X & X \\ \hline
				3-methylpentane & X & X & L & L & O & O & X & X \\ \hline
				2,2,4-trimethylpentane & X & X & L & L & O & O & O & O \\ \hline
			\end{tabular}
		\end{center} 
	\end{table}

	Sections \ref{sec:eta_sat} and \ref{sec:T293highP} present results for saturated liquid viscosities and compressed liquid viscosities, respectively. In both sections, the \textit{n}-alkane results are followed by the branched alkane results. Simulation results are compared with the REFPROP viscosity correlations and experimental data from the ThermoData Engine (TDE) database \cite{TDE}.
	
	\subsection{Saturated Liquid} \label{sec:eta_sat}
	
	\subsubsection{Normal alkanes}
	
	%\begin{enumerate}
	%	\item Ethane is exception where Mie potential significantly over-predicts viscosity
	%	\item Propane, butane, n-octane all see significant improvement with Mie and TAMie
	%	\item C12 has spurious results
	%\end{enumerate}
	
	%Figures:
	%
	%\begin{enumerate}
	%	\item Ethane
	%	\item C3, C4, C8
	%	\item C12, C16
	%\end{enumerate}
	
	Figure \ref{fig:Saturation_Ethane} compares the TraPPE (UA LJ 12-6), TraPPE-2 (AUA LJ 12-6), TAMie (AUA Mie 14-6), Potoff (UA Mie 16-6), and the Bayesian parameter sets for $n = 13$, $14$, $15$, and $16$.
	
	\begin{figure}[htb!]
		\centering
		\includegraphics[width=3.2in]{compare_force_fields_ethane.png}
		\caption{Saturated liquid viscosities for ethane. Colors/symbols denote different force fields.}
		\label{fig:Saturation_Ethane}
	\end{figure} 
	
	Figure \ref{fig:Saturation_C3_C4_C8} compares the TraPPE (UA LJ 12-6), Potoff (UA Mie 16-6), and TAMie (AUA Mie 14-6) saturated liquid viscosities for propane, \textit{n}-butane, and \textit{n}-octane. Similar to what has been demonstrated in previous studies, the TraPPE force field significantly under predicts $\eta_{\rm liq}^{\rm sat}$ (between 30 and 80 \%) with the deviation increasing with decreasing temperature. By contrast, the Potoff and TAMie force fields agree with the REFPROP values for these compounds to within 10 \%. While TAMie deviations increase near the triple point temperature of propane, Potoff deviations are nearly constant over the entire temperature range studied for each compound. 
	
	%By contrast, the Potoff and TAMie force fields agree with the REFPROP values for these compounds to within 10 \% over the entire temperature range studied (which includes the triple point for propane), and do not demonstrate a strong temperature dependence.
	
	\begin{figure}[htb!]
		\centering
		\includegraphics[width=3.2in]{compare_force_fields_short_normal.pdf}
		\caption{Saturated liquid viscosities for propane (\textit{n}-C$_{3}$), \textit{n}-butane (\textit{n}-C$_{4}$), and \textit{n}-octane (\textit{n}-C$_{8}$). Top panel compares simulation results with REFPROP correlations and TDE data. Bottom panel computes the percent deviation between the simulation and REFPROP values. Colors/symbols denote different force fields.}
		\label{fig:Saturation_C3_C4_C8}
	\end{figure} 
	
	Figure \ref{fig:Saturation_C12_C16_C22} compares the TraPPE, Potoff, and TAMie saturated liquid viscosities for \textit{n}-dodecane, \textit{n}-hexadecane, and \textit{n}-docosane. Although the TraPPE results for these compounds are similar to those observed in Figure \ref{fig:Saturation_C3_C4_C8} for smaller \textit{n}-alkanes, TAMie and Potoff demonstrate a stronger temperature dependence for these larger \textit{n}-alkanes. The cause of this trend is unclear but, since it is only observed for larger compounds, it is likely attributed to the torsional potential or the transferability of the CH$_2$ Mie $n$-6 parameters.  
	
	\begin{figure}[htb!]
		\centering
		\includegraphics[width=3.2in]{compare_force_fields_long_normal.pdf}
		\caption{Saturated liquid viscosities for \textit{n}-dodecane (\textit{n}-C$_{12}$), \textit{n}-hexadecane (\textit{n}-C$_{16}$), and \textit{n}-docosane (\textit{n}-C$_{22}$). Top panel compares simulation results with REFPROP correlations and TDE data. Bottom panel computes the percent deviation between the simulation and REFPROP values. Colors/symbols denote different force fields.}
		\label{fig:Saturation_C12_C16_C22}
	\end{figure} 
	
	\subsubsection{Branched alkanes}
	
	%\begin{enumerate}
	%	\item Mie potential provides less improvement in these cases
	%\end{enumerate}
	
	%Figures:
	%
	%\begin{enumerate}
	%	\item IC4, NEOC5
	%	\item IC5, IC8
	%	\item IC6, 23DMB, 3MP
	%\end{enumerate}
	
	%Unfortunately, TAMie does not have parameters for C and, therefore, 
	
	%Figures \ref{fig:Saturation_IC4_NEOC5} to \ref{fig:Saturation_IC6_23DMB_3MP} compare the saturated liquid viscosities for each force field and branched alkane studied. Figure \ref{fig:Saturation_IC4_NEOC5} presents results for the more spherical compounds, namely, 2-methylpropane and 2,2-dimethylpropane. Figure \ref{fig:Saturation_IC5_IC8} presents results for a short and long chain, namely, 2-methylbutane and 2,2,4-trimethylpentane. Figure \ref{fig:Saturation_IC6_23DMB_3MP} presents results for various hexane isomers, namely, 2-methylpentane, 2,3-dimethylbutane, and 3-methylpentane. Each compound was simulated using the TraPPE (UA LJ 12-6) and Potoff (UA Mie 16-6) force fields. However, only 2-methylpropane and 2,2-dimethylpropane were simulated with AUA4 (AUA LJ 12-6) while 2,2-dimethylpropane and 2,2,4-trimethylpentane were not simulated using the TAMie (AUA Mie 14-6) force field.
	%
	%From Figure \ref{fig:Saturation_IC4_NEOC5} to \ref{fig:Saturation_IC6_23DMB_3MP}, we see that the Potoff S/L force field is not as accurate for the simulated branched alkanes as for the normal alkanes. However, it still provides considerable improvement compared to the LJ 12-6 based models, i.e., TraPPE and AUA4. Surprisingly, the TAMie force field performs much worse for these branched alkanes. 
	%
	%%Figure \ref{fig:Saturation_IC4_NEOC5} compares the TraPPE (UA LJ 12-6), AUA4 (AUA LJ 12-6), and Potoff (UA Mie 16-6) saturated liquid viscosities for 2-methylpropane and 2,2-dimethylpropane.
	%
	%\begin{figure}[p!]
	%	\centering
	%	\includegraphics[width=3.2in]{compare_force_fields_IC4_neoC5.pdf}
	%	\caption{Saturated liquid viscosities for 2-methylpropane and 2,2-dimethylpropane. Colors/symbols denote different force fields.}
	%	\label{fig:Saturation_IC4_NEOC5}
	%\end{figure} 
	%
	%%Figure \ref{fig:Saturation_IC5_IC8} compares the TraPPE, Potoff, and TAMie saturated liquid viscosities for 2-methylbutane and 2,2,4-trimethylpentane.
	%
	%\begin{figure}[p!]
	%	\centering
	%	\includegraphics[width=3.2in]{compare_force_fields_IC5_IC8.pdf}
	%	\caption{Saturated liquid viscosities for 2-methylbutane and 2,2,4-trimethylpentane. Colors/symbols denote different force fields.}
	%	\label{fig:Saturation_IC5_IC8}
	%\end{figure} 
	%
	%%Figure \ref{fig:Saturation_IC6_23DMB_3MP} compares the TraPPE, Potoff, and TAMie saturated liquid viscosities for 2-methylbutane and 2,2,4-trimethylpentane.
	%
	%\begin{figure}[p!]
	%	\centering
	%	\includegraphics[width=3.2in]{empty_figure.jpg}
	%	\caption{Saturated liquid viscosities for 2-methylpentane, 2,3-dimethylbutane, and 3-methylpentane. Colors/symbols denote different force fields.}
	%	\label{fig:Saturation_IC6_23DMB_3MP}
	%\end{figure}
	
	Figures \ref{fig:Saturation_short_branched} and \ref{fig:Saturation_long_branched} compare the saturated liquid viscosities for each force field and branched alkane studied. Figures \ref{fig:Saturation_short_branched} and \ref{fig:Saturation_long_branched} present results for the compounds classified by Potoff as ``short'' and ``long'', respectively. Specifically, Figure \ref{fig:Saturation_short_branched} depicts 2-methylpropane, 2,2-dimethylpropane, 2-methylbutane, and 2,3-dimethylbutane, while Figure \ref{fig:Saturation_long_branched} contains 2-methylpentane, 3-methylpentane, and 2,2,4-trimethylpentane. Each compound was simulated using the TraPPE (UA LJ 12-6) and Potoff (UA Mie 16-6) force fields. However, only 2,2-dimethylpropane was simulated with AUA4 (AUA LJ 12-6) while 2,2-dimethylpropane and 2,2,4-trimethylpentane were not simulated using the TAMie (AUA Mie 14-6) force field.
	
	%From Figure \ref{fig:Saturation_IC4_NEOC5} to \ref{fig:Saturation_IC6_23DMB_3MP}, we see that the Potoff S/L force field is not as accurate for the simulated branched alkanes as for the normal alkanes. However, it still provides considerable improvement compared to the LJ 12-6 based models, i.e., TraPPE and AUA4. Surprisingly, the TAMie force field performs much worse for these branched alkanes. 
	
	\begin{figure}[htb!]
		\centering
		\includegraphics[width=3.2in]{compare_force_fields_short_branched.pdf}
		\caption{Saturated liquid viscosities for 2-methylpropane (\textit{i}-C$_{4}$), 2,2-dimethylpropane (neo-C$_{5}$), 2-methylbutane (\textit{i}-C$_{5}$), and 2,3-dimethylbutane (2,3-DMC$_4$). Top panel compares simulation results with REFPROP correlations and TDE data. Bottom panel computes the percent deviation between the simulation and REFPROP values. For clarity, values in top panel are shifted by $\Delta \eta$. Colors/symbols denote different force fields.}
		\label{fig:Saturation_short_branched}
	\end{figure} 
	
	\begin{figure}[htb!]
		\centering
		\includegraphics[width=3.2in]{compare_force_fields_long_branched.pdf}
		\caption{Saturated liquid viscosities for 2-methylpentane (\textit{i}-C$_{6}$), 3-methylpentane (3-MC$_5$), and 2,2,4-trimethylpentane (\textit{i}-C$_{8}$). Top panel compares simulation results with REFPROP correlations and TDE data. Bottom panel computes the percent deviation between the simulation and REFPROP values. For clarity, values in top panel are shifted by $\Delta \eta$. Colors/symbols denote different force fields.}
		\label{fig:Saturation_long_branched}
	\end{figure} 
	
	From Figures \ref{fig:Saturation_short_branched} and \ref{fig:Saturation_long_branched}, we see that the Potoff S/L and TAMie force fields are not as accurate for these branched alkanes as for the normal alkanes. In particular, Potoff and TAMie demonstrate the same temperature dependence observed for other force fields, where the deviations are largest at lower temperatures. However, Potoff still provides considerable improvement compared to the LJ 12-6 based models, i.e., TraPPE and AUA4. Note that the performance is similar for the Potoff ``short'' and ``long'' parameters in Figures \ref{fig:Saturation_short_branched} and \ref{fig:Saturation_long_branched}, respectively. 
	
	%The one apparent exception is 3-methylpentane, where the Potoff deviations are less than 10 \% over the entire temperature range. However, considering the availabi 
	
	The deviations for each force field are largest for 2-methylpropane and 2,2-dimethylpropane. Since these compounds are primarily composed of CH$_3$ UA sites, this poor performance is likely due to the assumption that the CH$_3$ non-bonded parameters are transferable from \textit{n}-alkanes to branched alkanes. Improvement might be possible if the CH$_3$ parameters were different depending on the neighboring UA site type. However, we emphasize that REFPROP states that the viscosity correlation for 2,2-dimethylpropane is not of ``reference quality.''
	
	\subsection{Compressed liquid} \label{sec:T293highP}
	
	Section \ref{sec:eta_sat} demonstrates that Mie $n$-6 based force fields (Potoff and TAMie) are considerably more reliable for predicting saturated liquid viscosities than LJ 12-6 based force fields (TraPPE and AUA4). However, Reference \citenum{Postdoc_2} confirms that the Mie $n$-6 potential is too repulsive at short distances for $n > 12$, which causes the Potoff (16-6) and TAMie (14-6) force fields to over estimate pressure at high densities. Since viscosity increases with a more repulsive potential (increasing $n$), our \textit{ansatz} was that Potoff and TAMie would over estimate $\eta$ at high densities/pressures. 
	
	Surprisingly, Reference \citenum{Gordon2006} shows that a (modified) Mie 14-6 (CH$_3$) and 20-6 (CH$_2$) potential can accurately predict the $\eta$-$P$ dependence for \textit{n}-hexadecane at 400 MPa. Note that Reference \citenum{Gordon2006} did not report the $\eta$-$\rho$ dependence, utilizes a slightly modified Mie formulation, and included saturated liquid viscosity data in the non-bonded parameterization. The purpose of this section is to determine if the Potoff 16-6 and TAMie 14-6 force fields are also reliable for estimating the $\eta$-$P$ dependence. To provide additional insight into the consequences of using a Mie potential with $n > 12$, we present results for the $\eta$-$\rho$ dependence as well. 
	
%	Note that both the Potoff and TAMie non-bonded potentials use $n > 12$.
	
	% Note that Reference \citenum{Gordon2006} utilized a slightly modified version of the Mie potential
	
	%The purpose of this section is to determine if a similar phenomenon is observed for viscosity estimates at high densities/pressures along the isotherm of 293 K.
	
	%The purpose of this section is to determine if a similar phenomenon is observed for viscosity estimates at high densities/pressures along the isotherm of 293 K. 
	
	\subsubsection{Normal alkanes}
	
%	\begin{enumerate}
%		\item Propane has accurate viscosity-P but not viscosity-rho
%		\item Butane appears to agree more closely with recent REFPROP correlation
%	\end{enumerate}
	
	%Figures:
	%
	%\begin{enumerate}
	%	\item Propane $\eta-\rho$ $\eta-P$
	%	\item Butane $\eta-\rho$ $\eta-P$
	%	\item n-Octane $\eta-\rho$ $\eta-P$
	%	%	\item n-Dodecane $\eta-\rho$ $\eta-P$?
	%\end{enumerate}
	
	Figures \ref{fig:T293highP_C3}, \ref{fig:T293highP_C4}, and \ref{fig:T293highP_C8} compare the elevated pressure viscosities for propane, \textit{n}-butane, and \textit{n}-octane, respectively. Each compound is simulated using the TraPPE, Potoff, and TAMie force fields at four or five densities. Simulation results are compared with REFPROP correlations and TDE data, when available. All TDE data for temperatures between 288 and 298 K are included. REFPROP uncertainties are assumed to be a constant percent deviation as reported in the corresponding publication. ``REFPROP extrapolation'' are values that lie outside of the ``reference quality'' range. This extrapolation is intended to guide the eye when comparing with simulation results at high densities/pressures.
	
	% Note that for propane and \textit{n}-butane (Figures \ref{fig:T293highP_C3} and \ref{fig:T293highP_C4}) each force field is simulated at the same density, while for \textit{n}-octane (Figure \ref{fig:T293highP_C8}) the force fields are simulated at the same pressure. 
	
	%As the REFPROP viscosity correlation is not recommended above 100 MPa at 293 K, we have plotted some experimental data to guide the eye.
	
	\begin{figure}[htb!]
		\centering
		\includegraphics[width=6.4in]{compare_REFPROP_T293highP_C3H8.pdf}
		\caption{Compressed liquid viscosities at 293 K for propane. Top panels provide $\eta$-$\rho$ and $\eta$-$P$ dependence. Bottom panels present deviations between simulated $(\eta_{\rm sim})$ and REFPROP $(\eta_{\rm REFPROP})$ values with respect to $\rho$ and $P$. Dashed lines correspond to REFPROP uncertainties. Dotted lines are extrapolation values outside of ``reference quality'' range. Colors/symbols denote different force fields and experimental data. Simulation uncertainties are obtained from bootstrap re-sampling and are presented at the 95 \% confidence level.}
		\label{fig:T293highP_C3}
	\end{figure} 
	
	Figure \ref{fig:T293highP_C3} demonstrates that the TraPPE force field has a constant negative bias even with increasing density/pressure. The TAMie force field has the most accurate $\eta$-$\rho$ dependence, i.e., the error does not increase with respect to density. By contrast, the Potoff potential demonstrates considerable over estimation of $\eta$ at high densities, which is likely attributed to the overly repulsive Mie 16-6 potential at close distances. Remarkably, the Potoff force field is the most accurate at predicting the $\eta$-$P$ dependence from saturation pressure to 1 GPa. This can be explained as a cancellation of errors since the Potoff force field significantly over predicts both viscosity and pressure at high densities. Note that the increase in Potoff deviations for the two highest pressures can potentially be explained by the large uncertainty in the REFPROP correlation and extrapolation at these extreme pressures.
	
	\begin{figure}[htb!]
		\centering
		\includegraphics[width=6.4in]{compare_REFPROP_T293highP_C4H10.pdf}
		\caption{Compressed liquid viscosities at 293 K for \textit{n}-butane. See caption of Figure \ref{fig:T293highP_C3}.}
		\label{fig:T293highP_C4}
	\end{figure} 
	
	\begin{figure}[htb!]
		\centering
		\includegraphics[width=6.4in]{compare_REFPROP_T293highP_C8H18.pdf}
		\caption{Compressed liquid viscosities at 293 K for \textit{n}-octane. See caption of Figure \ref{fig:T293highP_C3}.}
		\label{fig:T293highP_C8}
	\end{figure} 
	
	The results in Figures \ref{fig:T293highP_C4} and \ref{fig:T293highP_C8} for \textit{n}-butane and \textit{n}-octane, respectively, are similar to those in Figure \ref{fig:T293highP_C3} for propane. Specifically, the TraPPE force field under predicts $\eta$ at all densities/pressures, the TAMie force field provides the most accurate $\eta$-$\rho$ dependence, while the Potoff force field over predicts $\eta$ with respect to $\rho$ but accurately predicts the $\eta$-$P$ trend. We conclude that the Potoff force field is overly repulsive at short distances, and should not be used to estimate the $\eta$-$\rho$ dependence. However, the Potoff force field is the most reliable for estimating the $\eta$-$P$ dependence, which is often the desired relationship in practice.  
	
	\subsubsection{Branched alkanes}
	
%	\begin{enumerate}
%		\item Similar to n-alkanes? 
%		\item Wrong torsions matters?
%	\end{enumerate}
	
	%Figures:
	%
	%\begin{enumerate}
	%	\item Isobutane $\eta-\rho$ $\eta-P$
	%	\item Isopentane $\eta-\rho$ $\eta-P$
	%	%	\item Isohexane $\eta-\rho$ $\eta-P$?
	%	\item Isooctane $\eta-\rho$ $\eta-P$
	%	%	\item Neopentane $\eta-\rho$ $\eta-P$?
	%	\item 3-methylpentane $\eta-\rho$ $\eta-P$?
	%	%	\item 2,3-dimethylbutane $\eta-\rho$ $\eta-P$?
	%\end{enumerate}
	
   The trends observed in Figures \ref{fig:T293highP_IC4} to \ref{fig:T293highP_IC8} are consistent with the compressed liquid trends for \textit{n}-alkanes. Specifically, TraPPE under-predicts the viscosity with respect to both $\rho$ and $P$. Potoff over-predicts $\eta$ with respect to $\rho$ but provides a reasonable estimate of the $\eta$-$P$ trend. As observed previously in Section \ref{sec:eta_sat}, Potoff and TAMie are less accurate for branched alkanes than for \textit{n}-alkanes. In particular, the Potoff $\eta$-$P$ trends are systematically lower than the REFPROP correlations for 2-methylbutane and 3-methylpentane. However, note that the Potoff $\eta$-$P$ trends are more reliable for 2-methylpropane and 2,2,4-trimethylpentane. These results cannot be attributed to the ``short'' or ``long'' parameter distinction. 
	
	\begin{figure}[htb!]
		\centering
		\includegraphics[width=6.4in]{compare_REFPROP_T293highP_IC4H10.pdf}
		\caption{Compressed liquid viscosities at 293 K for 2-methylpropane. See caption of Figure \ref{fig:T293highP_C3}.}
		\label{fig:T293highP_IC4}
	\end{figure} 
	
	\begin{figure}[htb!]
		\centering
		\includegraphics[width=6.4in]{compare_REFPROP_T293highP_IC5H12.pdf}
		\caption{Compressed liquid viscosities at 293 K for 2-methylbutane. See caption of Figure \ref{fig:T293highP_C3}.}
		\label{fig:T293highP_IC5}
	\end{figure} 
	
	\begin{figure}[htb!]
		\centering
		\includegraphics[width=6.4in]{compare_REFPROP_T293highP_3MPentane.pdf}
		\caption{Compressed liquid viscosities at 293 K for 3-methylpentane. See caption of Figure \ref{fig:T293highP_C3}.}
		\label{fig:T293highP_3MP}
	\end{figure} 
	
	\begin{figure}[htb!]
		\centering
		\includegraphics[width=6.4in]{compare_REFPROP_T293highP_IC8H18_without_devPlots.pdf}
		\caption{Compressed liquid viscosities at 293 K for 2,2,4-trimethylpentane. See caption of Figure \ref{fig:T293highP_C3}.}
		\label{fig:T293highP_IC8}
	\end{figure} 
	
	\section{Discussion/Limitations} \label{Discussion/Limitations}
	
	While the Potoff force field significantly over-predicts the $\eta$-$\rho$ dependence at $T= 293$ K, it does not over-predict $\eta$ for the highest saturated liquid densities (those near the triple point temperature) (cf. Figures \ref{fig:Saturation_C3_C4_C8} and \ref{fig:T293highP_C3}). To better understand this, Figure \ref{fig:RDF_comparison_CH3} compares the radial distribution functions (RDF) for three different state points, namely, near the triple point $(T=86$ K and $\rho = 732.63$ kg/m$^3)$ and two densities along the $T = 293$ K isotherm ($\rho = 732.63$ kg/m$^3$ and $\rho = 806.23$ kg/m$^3)$. Note that, in order to provide a fair comparison between force fields, the RDF is plotted with respect to a reduced distance, namely, $r/r_{\rm min}$.
	
	% $(T_{\rm tp}$ and $\rho_{\rm tp})$ and two densities along the $T = 293$ K isotherm ($\rho_{\rm tp}$ and the maximum $\rho$)
	
	\begin{figure}[htb!]
		\centering
		\includegraphics[width=3.2in]{RDF_comparison_CH3.pdf}
		\caption{Comparison of radial distribution function $(g(r))$ for CH$_3$-CH$_3$ interactions. The top panel compares two different temperatures near the triple point isochore. The bottom panel compares two different densities along the 293 K isotherm. Colors correspond to different temperatures while line styles denote different force fields.}
		\label{fig:RDF_comparison_CH3}
	\end{figure} 
	
	The top panel of Figure \ref{fig:RDF_comparison_CH3} demonstrates that the RDF is shifted to the left (closer interactions) when increasing the temperature from 86 K to 293 K. Note that, although the magnitude of this shift is similar for all three force fields, the Potoff viscosities are impacted the most from this shift due to the steepness of the Mie 16-6 potential. By contrast, the bottom panel demonstrates that increasing the density at constant temperature does not result in a shift but does increase the frequency of close-range interactions. Therefore, the overly repulsive Mie 16-6 potential is only problematic at high densities if there is sufficient thermal energy for the system to sample extremely close-range interactions. This explains why the Potoff force field is reliable at high densities along the saturation curve but over-estimates the $\eta$-$\rho$ dependence at 293 K.	
	
	%Therefore, It is clear from Figure \ref{fig:RDF_comparison_CH3} that the higher pressure simulations sample from much closer interactions than the triple point and, therefore, these close interactions cause the over-estimation of compressed liquid viscosities but not saturated high density liquid viscosities.
	
	% This effect is due to the low thermal energy available near the triple point temperature. 
	
%	\begin{enumerate}
%		\item Finite-size effects
%		\item Fixed vs flexible bonds
%		\item Simulation time, high viscosities
%		\item Cut-offs for C12
%		\item RDFs for propane
%	\end{enumerate}
%	
%	\begin{enumerate}
%		\item Discussion
%		\begin{enumerate}
%			\item Mie potentials parameterized with VLE data provide significant improvement over LJ 12-6
%			\item Potoff over-predicts $\eta-\rho$ dependence while TAMie is fairly accurate
%			\item Potoff appears to be slightly more accurate for $\eta-P$
%			\item Branched alkanes are not as accurate, perhaps assumption of transferability or torsional parameters
%		\end{enumerate}
%		\item Limitations
%		\begin{enumerate}
%			\item Largest viscosity simulations are slow to converge and unclear if simulations are sufficiently long
%			\item Tail-corrections could impact dynamics
%			\item Using REFPROP saturation conditions instead of force fields
%		\end{enumerate}
%	\end{enumerate}

	\subsection{Finite-size effects}

	Previous studies have demonstrated that finite-size effects are often negligible for viscosity estimates with equilibrium molecular dynamics. However, since this analysis is typically not reported in the literature, the ``Best Practices'' guide recommends validating that finite-size effects are indeed negligible. For this reason, Figure \ref{fig:finite_size_effects} compares the results for simulations performed with 100, 200, 400, and 800 molecules of propane.   
	
	\begin{figure}[htb!]
		\centering
		\includegraphics[width=3.2in]{empty_figure.jpg}
		\caption{Finite-size effects. Five different }
		\label{fig:finite_size_effects}
	\end{figure} 
	
	Notice that the averages and uncertainties typically overlap considerably and that there is no clear trend with respect to $N^{-1/3}$. In addition to this visual inspection, we also perform a ``random permutation test'' to determine that the results from the different system sizes are statistically indistinguishable. The algorithm for ``random permutation testing'' is the following:
	
	\begin{enumerate}
		\item Compute the ``observed'' sum-of-squares between the four different system sizes $(SS_{\rm obs})$
		\item Randomly divide the replicate simulations of each system size into four groups, ``A'', ``B'', ``C'', and ``D'' \label{item:shuffling}
		\item Average the replicate Green-Kubo integrals of each group \label{item:averaging}
		\item Fit Equation \ref{eq: Double exponential} to the averages from Step \ref{item:averaging} \label{item:fitting}
		\item Calculate the ``permutated'' sum-of-squares between the four different groups $(SS_{\rm perm}$ \label{item:SS_ABCD}
		\item Repeat Steps \ref{item:shuffling} to \ref{item:SS_ABCD} hundreds of times $(N_{\rm perm})$
		\item Count the number of permutations with $SS_{\rm perm} > SS_{\rm obs}$ $(N_{\rm count})$ \label{item:counting}
		\item Assign a $p$-value equal to the ratio of $N_{\rm count}$ divided by $N_{\rm perm}$
		\item If $p > 0.05$, we fail to reject the Null-hypothesis that all four system sizes are statistically equivalent 
	\end{enumerate}
	
	%	\begin{enumerate}
	%		\item Compute the ``observed'' $F$-statistic between the four different system sizes $(F_{\rm obs})$
	%		\item Randomly divide the replicate simulations of each system size into two groups, ``A'' and ``B'' \label{item:shuffling}
	%		\item Average the replicate Green-Kubo integrals of both groups \label{item:averaging}
	%		\item Fit Equation \ref{eq: Double exponential} to the averages from Step \ref{item:averaging} \label{item:fitting}
	%		\item Calculate the $F$-statistic for the two different groups $(F_{\rm AB}$ \label{item:F_AB}
	%		\item Repeat Steps \ref{item:shuffling} to \ref{item:F_AB} hundreds of times $(N_{\rm perm})$
	%		\item Count the number of permutations with $F_{\rm AB} > F_{\rm obs}$ \label{item:counting}
	%		\item Divide the value obtained in Step \ref{item:counting} by $N_{\rm perm}$ 
	%	\end{enumerate}
	
	This approach is employed in place of the standard Analysis of Variance (ANOVA) test because it is not possible to compute a reliable $F$-statistic. Computing an $F$-statistic requires computing the sum-of-squares between replicates, but an individual replicate simulation does not provide a meaningful estimate of $\eta$. Therefore, we can only compute the sum-of-squares between averages and not the sum-of-squares between replicates. However, since the sum-of-squares between replicates is a constant value regardless of the permutations, comparing the sum-of-squares is equivalent to comparing the $F$-statistic.  
	
	%	\begin{enumerate}
	%		\item Computing the difference between two different system size
	%		\item Randomly dividing the replicate simulations of each system size into two groups, ``A'' and ``B''
	%		\item Averaging the replicate Green-Kubo integrals of both groups \label{step:averaging}
	%		\item Fitting Equation \ref{eq: Double exponential} to the averages from Step \ref{step:averaging}
	%		\item Calculating the difference between $\eta^{\infty}_{\rm A}$ and $\eta^{\infty}_{\rm B}$ $(\Delta \eta_{\rm AB})$
	%		\item Repeating steps 2 to 5 hundreds of times
	%		\item Binning the distribution of $\Delta \eta_{\rm AB}$ values
	%		\item Comparing the 
	%	\end{enumerate}
	
	%	 randomly dividing the replicate simulations of each system size into two groups, ``A'' and ``B,'' computing the average Green-Kubo integral for both groups, fitting Equation \ref{eq: Double exponential} for both averages, calculating the difference between $\eta^{\infty}_{\rm A}$ and $\eta^{\infty}_{\rm B}$, repeating this process hundreds of times, and  selecting a subset of replicate simulations from each system size and computing the average.
	
	%	\begin{enumerate}
	%		\item Simulation results for 100, 200, 400, and 800 molecules
	%	\end{enumerate}
	
	\subsection{Cut-off distance}
	
	The choice of cut-off distance is a subtle but important decision. In this study, we implement a 1.4 nm cut-off for each force field except Potoff, which utilizes only a 1.0 nm cut-off. This choice was made because the Potoff force field was parameterized with a 1.0 nm cut-off. 
	
	The ``Best Practices'' guide suggests that cut-off lengths could be significant but does not provide any convincing evidence to prove or disprove this notion. To address this issue, we perform simulations of the Potoff force field using three different cut-off distances. Specifically, Figure \ref{fig:cutoff_distance} presents the Potoff viscosity values for propane, \textit{n}-butane, \textit{n}-octane, and \textit{n}-dodecane using cut-offs of 1.0 nm, 1.4 nm, 1.8 nm. 
	
	\begin{figure}[htb!]
		\centering
		\includegraphics[width=3.2in]{empty_figure.jpg}
		\caption{Impact of cut-off distance.}
		\label{fig:cutoff_distance}
	\end{figure} 
	
	Figure \ref{fig:cutoff_distance} demonstrates that for smaller compounds the impact of cut-off is negligible, while a 1.0 nm cut-off causes a significant error for \textit{n}-dodecane. For this reason, the Potoff results presented previously in Figure \ref{fig:Saturation_C12_C16} were obtained using a 1.4 nm cut-off. In fact, \textit{n}-hexadecane and \textit{n}-docosane were unstable with a 1.0 nm cut-off and a 2 fs time-step. Reducing the time step to 1 fs was capable of stabilizing the 1.0 nm cut-off. 
	
	The instability of a 1.0 nm cut-off (with 2 fs time-steps) demonstrates the importance of verifying that the cut-off distance is long enough to not impact the system dynamics. Furthermore, this demonstrates why alternative tail modifications are ideal for molecular dynamics, e.g., force-shift and switch-force. Unfortunately, these tail modifications significantly impact saturation properties, suggesting that the non-bonded parameters must be re-optimized with the modified tail. Therefore, performing simulations with a force-shift or switch-force Potoff, TraPPE, TAMie, and AUA4 potential would likely result in inaccurate viscosities. Re-parameterizing the non-bonded interactions for a force-shift or switch-force potential is beyond the scope of this study.
	
	\subsection{$\rho_{\rm liq}^{\rm sat}$}
	
	There are at least three reasons why we perform simulations at the REFPROP $\rho_{\rm liq}^{\rm sat}$ instead of the force field $\rho_{\rm liq}^{\rm sat}$. First, this approach allows for a fair comparison of the force fields' ability to predict viscosity, without penalizing force fields which are less accurate at predicting $\rho_{\rm liq}^{\rm sat}$ or rewarding force fields that mask their deficiencies in predicting viscosity by over- or under-estimating $\rho_{\rm liq}^{\rm sat}$. Second, since each of the studied force fields utilized $\rho_{\rm liq}^{\rm sat}$ data in their optimization, deviations between the REFPROP and force field values are small, typically less than 1 \%. However, small differences in density have been reported to result in large differences in viscosity. For this reason, a small set of validation simulations are performed to determine the variability caused by utilizing the REFPROP densities. The force field saturated liquid densities were obtained from the literature.      
	
	The use of REFPROP $\rho_{\rm liq}^{\rm sat}$ caused some simulations to be in a meta-stable state. Specifically, this occurs when the force field vapor pressure is less than the REFPROP vapor pressure. Fortunately, this is uncommon as Potoff, TAMie, and AUA4 are quite reliable for estimating $P_{\rm vap}^{\rm sat}$ and TraPPE significantly over-estimates $P_{\rm vap}^{\rm sat}$.
	
	\section{Conclusions} \label{Conclusions}
	
	This study demonstrates the improvement that has taken place over the past two decades for predicting viscosity with molecular simulation. First, the ``Best Practices'' for EMD lead to more reproducible results. Second, the state-of-the-art Mie n-6 force fields are significantly more accurate than the traditional Lennard-Jones 12-6 force fields. More specifically, the Potoff and TAMie force fields typically predict saturated liquid viscosities for \textit{n}-alkanes to within 10 \% of the REFPROP values. By contrast, the TraPPE and AUA4 models under-predict saturated liquid viscosities by 30 \% to 50 \%, where the deviations are largest at lower temperatures. While Potoff and TAMie are also more reliable for branched alkanes, deviations are larger and demonstrate a similar temperature dependence. The key limitation of the Potoff force field is that the choice of n$=16$ is too repulsive at high densities, which causes the viscosity to be over-estimated at high densities. Due to a fortuitous cancellation of errors, the Potoff potential does provide a reliable $\eta$-$P$ trend. Since TAMie uses n$=14$, the $\eta$-$\rho$ trend is slightly more reliable than that of Potoff. It is important to emphasize that transport properties were not included in the training set for parameterizing the Potoff and TAMie force fields. Therefore, the results from this study demonstrate that the improved prediction of static vapor-liquid coexistence properties obtained with Mie n-6 potentials also results in improved prediction of a transport property, namely, liquid viscosity.
	
	\section*{Acknowledgments}
	
	We are grateful for the internal review provided by NIST BERB Reviewer 1 and NIST BERB Reviewer 2 from the National Institute of Standards and Technology (NIST). 
	
	This research was performed while Richard A. Messerly held a National Research Council (NRC) Postdoctoral Research Associateship at NIST and while Michelle C. Anderson held a Summer Undergraduate Research Fellowship (SURF) position at NIST.
	
	\bibliographystyle{unsrt}
	\bibliography{Special_issue_references}
	
	\section{Supporting Information}
	
	\subsection{Gromacs input files}
	
	We have provided example input files for simulating \textit{n}-isooctane at BLANK K with the Potoff force field in GROMACS (see attached .gro, .top, and .mdp files). Additionally, all files that were used to generate the results from this study can be found that the GitHub repository www.github.com/ramess101/IFPSC\_10.
	
%	\begin{enumerate}
%		\item Include all the .gro files
%		\item Include all the .top file templates
%		\item Include .mdp files
%		\item Or we can just include an example and then refer them to the GitHub website
%	\end{enumerate}
%	
	\subsection{Tabulated values}
	
	\begin{enumerate}
		\item Ethane
		\begin{enumerate}
			\item Saturation
			\begin{enumerate}
				\item Potoff
				\item TraPPE
				\item AUA4
				\item TAMie
			\end{enumerate}
			\item T293 highP
			\begin{enumerate}
				\item Potoff
				\item TraPPE
				\item AUA4
				\item TAMie
			\end{enumerate}
		\end{enumerate}
		\item Propane
		\begin{enumerate}
			\item Saturation
			\begin{enumerate}
				\item Potoff
				\item TraPPE
				\item AUA4
				\item TAMie
			\end{enumerate}
			\item T293 highP
			\begin{enumerate}
				\item Potoff
				\item TraPPE
				\item AUA4
				\item TAMie
			\end{enumerate}
		\end{enumerate}
		\item n-Butane
		\begin{enumerate}
			\item Saturation
			\begin{enumerate}
				\item Potoff
				\item TraPPE
				\item AUA4
				\item TAMie
			\end{enumerate}
			\item T293 highP
			\begin{enumerate}
				\item Potoff
				\item TraPPE
				\item AUA4
				\item TAMie
			\end{enumerate}
		\end{enumerate}
		Repeat for all other compounds with corresponding potentials    
	\end{enumerate}
		
	\subsection{Simulation length effects}
	
	For less viscous systems, i.e., saturation and low pressures, a 1 ns simulation is typically sufficient for the Green-Kubo integral to reach a plateau. However, even when an apparent plateau is observed, too short of simulations can lead to systematic bias in $\eta$. For this reason, as recommended by ``Best Practices,'' we verify that the estimated viscosity obtained from a 1 ns trajectory is consistent with that obtained from 2 ns, 4 ns, and 8 ns simulations.
	
	\begin{figure}[htb!]
		\centering
		\includegraphics[width=3.2in]{empty_figure.jpg}
		\caption{Finite-size effects.}
		\label{fig:simulation_time}
	\end{figure} 

    For more viscous systems, i.e., greater than 100 MPa, a 1 ns simulation is too short to observe a plateau in the Green-Kubo integral. In these cases, we increased the simulation time to a value between 2 and 8 ns. Due to the increased computational cost of such simulations, we did not perform an exhaustive test with increasing simulation time. Instead, the choice of simulation time was determined primarily by the ability to detect a plateau region. Therefore, it is possible that even longer simulations are required for the most viscous systems.
    
    Due to inherently slow dynamics in highly viscous systems (greater than 0.02 Pa-s), obtaining well-converged Green-Kubo integrals is extremely challenging. Additional replicate simulations can help reduce noise.
    
    Storage limitations become a concern with simulations longer than 8 ns due to the frequency at which data are output (every 6 fs). Reading in these files for GROMACS to evaluate the data nearly crippled our computing cluster.
    
    Obtaining well-converged Green-Kubo integrals for highly viscous systems (greater than 0.02 Pa-s) is challenging and requires longer simulations.
	
%	\begin{enumerate}
%		\item Verified that 1 ns is long enough for larger compounds
%	\end{enumerate}
	
	\subsection{Validation Runs}
	
    To validate our methodology, we attempt to replicate viscosity estimates available on the NIST Reference Simulation Data website for TraPPE-UA ethane as well as literature values for TraPPE-UA \textit{n}-octane. Figure \ref{fig:validation_runs} compares the ethane and \textit{n}-octane results from this study with those from NIST and the literature, respectively.   
    
    \begin{figure}[htb!]
    	\centering
    	\includegraphics[width=3.2in]{empty_figure.jpg}
    	\caption{Comparison with NIST Reference Simulation Data and Reference \citenum{BLANK}.}
    	\label{fig:validation_runs}
    \end{figure} 
    
    The \textit{n}-octane validation is somewhat more useful than the ethane validation for at least three reasons. First, \textit{n}-octane includes angle and torsional contributions that are absent in ethane. Second, the literature provides values for both rigid and flexible bonds. Third, the \textit{n}-octane results are for fixed elevated pressures, which provides validation of our $NPT$ ensemble results.
    
%	
%	\begin{enumerate}
%		\item Ethane NIST
%		\item n-Octane Literature
%	\end{enumerate}
	
	\subsection{Bond types, Harmonic vs LINCS}
	
    To test how sensitive the results presented in Section \ref{Results} are to the use of fixed bond-lengths, we perform additional simulations with flexible bonds. Specifically, we use the traditional harmonic bond potential:
    \begin{equation} \label{eq:harmonic_bond}
    u^{\rm bond} = \frac{k_{\rm b}}{2} \left(r-r_{\rm eq}\right)^2
    \end{equation}
    where $u^{\rm bond}$ is the bonded potential, $k_{\rm b}$ is the harmonic force constant, and $r_{\rm eq}$ is the equilibrium bond-length. To determine the impact, we perform simulations with two different values of $k_{\rm b}$, namely, BLANK (taken from reference BLANK) and BLANK (an arbitrarily large value). Figure \ref{fig:fixed_flexible} demonstrates that the difference between fixed and flexible bonds is negligible in certain cases, but for the larger force constant a systematic increase in viscosity is observed. Bootstrap resampling confirms that the LINCS and $k_{\rm b} = $ BLANK results are statistically indistinguishable, while the $k_{\rm b} = $ BLANK is statistically different.

	\begin{figure}[htb!]
		\centering
		\includegraphics[width=3.2in]{empty_figure.jpg}
		\caption{Fixed bond-lengths compared with two different harmonic bond potentials.}
		\label{fig:fixed_flexible}
	\end{figure} 
	
%	\begin{enumerate}
%		\item Propane and n-butane with harmonic (arbirary bond constant) shows systematic increase
%	\end{enumerate}
	
	\subsection{Green-Kubo analysis}
	
	This section provides a detailed example of how we obtain estimates for $\eta$ with its corresponding uncertainty. Figure BLANK depicts a typical autocorrelation function obtained by executing the GROMACS ``energy --vis.'' By default, GROMACS partitions the complete simulation into twelve evenly sized time blocks. Therefore, the autocorrelation in Figure BLANK is the average of twelve different time origins. GROMACS then performs a simple trapezoidal integration of neighboring points to obtain the Green-Kubo integral. The Green-Kubo integral with respect to time is output in the ``visco.xvg'' file. Figure BLANK presents the Green-Kubo integral from forty replicate simulations. Although a single replicate is often quite noisy at long times, the average of these replicates converges smoothly (see Figure BLANK). Figure BLANK shows that the fluctuations, or standard deviation, increases with time but is adequately modeled with $A t^{b}$. The line labeled ``cut-off'' in Figures BLANK and BLANK is the time at which $\sigma_{\eta} \approx \eta^\infty$. Data beyond this time are excluded from the fit of the double-exponential function. Bootstrap resampling provides an estimate of the uncertainty. Figure BLANK shows that, typically, the bootstrapped distribution is quite normal. The line labeled ``bootstraps'' in Figure BLANK are the lower and upper 95 \% confidence interval.

    The results depicted in Figure BLANK are from BLANK, BLANK, BLANK.
	
	\begin{figure}[htb!]
		\centering
		\includegraphics[width=3.2in]{empty_figure.jpg}
		\caption{Autocorrelation function with respect to time.}
		\label{fig:autocorrelation}
	\end{figure} 

	\begin{figure}[htb!]
		\centering
		\includegraphics[width=3.2in]{empty_figure.jpg}
		\caption{Replicate simulations, average, fit to average, cut-off, and bootstrap uncertainties.}
		\label{fig:replicates}
	\end{figure} 

	\begin{figure}[htb!]
		\centering
		\includegraphics[width=3.2in]{empty_figure.jpg}
		\caption{Standard deviation of replicate simulations with respect to time.}
		\label{fig:standard_deviation}
	\end{figure} 

	\begin{figure}[htb!]
		\centering
		\includegraphics[width=3.2in]{empty_figure.jpg}
		\caption{Bootstrap distribution of $\eta$.}
		\label{fig:bootstraps}
	\end{figure} 

	
%	\begin{enumerate}
%		\item Raw data, i.e., multiple replicates with the average
%		\item Exclude low time data and have a heuristic for determining the cut-off time
%	\end{enumerate}
%	
%	Example analysis, i.e., bootstrap distribution, replicates
	
	\subsection{MCMC?}
	
\end{document}
